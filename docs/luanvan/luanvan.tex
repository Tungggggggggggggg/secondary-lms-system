\documentclass[12pt, a4paper]{report}

% --- UNIVERSAL PREAMBLE BLOCK FOR VIETNAMESE ---
\usepackage[a4paper, top=20mm, bottom=20mm, left=35mm, right=20mm]{geometry}
\usepackage{fontspec}
\usepackage[bidi=default, provide=*]{babel}

% Cấu hình ngôn ngữ và font chữ (Sử dụng Noto Serif để thay thế Times New Roman trong môi trường này)
\babelprovide[main, import]{vietnamese}
\babelprovide[import]{english}


\IfFontExistsTF{Times New Roman}{
\babelfont{rm}{Times New Roman}
\babelfont[vietnamese]{rm}{Times New Roman}
}{
\babelfont{rm}{Latin Modern Roman}
\babelfont[vietnamese]{rm}{Latin Modern Roman}
}


% Fix cho danh sách khi sử dụng ngôn ngữ không phải tiếng Anh
\usepackage{enumitem}
\setlist[itemize]{label=-}
% --- GÓI BỔ TRỢ HỌC THUẬT ---
\usepackage{amsmath, amsfonts, amssymb}
\usepackage{graphicx}
\graphicspath{{docs/luanvan/}{docs/luanvan/figures/}{docs/luanvan/figures/uml/}{figures/}{figures/uml/}{uml/}{out/docs/luanvan/figures/}{out/docs/luanvan/figures/uml/}{out/docs/luanvan/figures/uml/usecase-overall/}{out/docs/luanvan/figures/uml/usecase-anti-cheat/}{out/docs/luanvan/figures/uml/sequence-exam-events/}{out/docs/luanvan/figures/uml/activity-suspicious-behavior/}{out/docs/luanvan/figures/uml/context-eduverse/}{out/docs/luanvan/figures/uml/component-architecture/}{out/docs/luanvan/figures/uml/db-core/}}
\usepackage{setspace}
\usepackage{indentfirst}
\usepackage{titlesec}
\usepackage{listings}
\usepackage[table]{xcolor}
\usepackage{booktabs}
\usepackage{array}
\usepackage{tabularx}
\usepackage{caption}
\usepackage{float}
\usepackage{needspace}

% Cấu hình giãn dòng chuẩn luận văn (1.5 line spacing)
\onehalfspacing
\setlength{\parindent}{1.27cm}
\setlength{\emergencystretch}{2em}

% Cấu hình mã nguồn (Code) chuyên nghiệp
\definecolor{codegreen}{rgb}{0,0.6,0}
\definecolor{codegray}{rgb}{0.5,0.5,0.5}
\definecolor{codepurple}{rgb}{0.58,0,0.82}
\definecolor{backcolour}{rgb}{0.95,0.95,0.92}

\lstdefinelanguage{JavaScript}{
    keywords={await,break,case,catch,class,const,continue,debugger,default,delete,do,else,export,extends,false,finally,for,function,if,import,in,instanceof,let,new,null,return,super,switch,this,throw,true,try,typeof,var,void,while,with,yield},
    keywordstyle=\color{magenta},
    sensitive=true,
    morecomment=[l]{//},
    morecomment=[s]{/*}{*/},
    morestring=[b]',
    morestring=[b]",
}

\lstdefinelanguage{TypeScript}[]{JavaScript}{
    keywords={any,as,boolean,constructor,declare,enum,get,module,namespace,number,private,protected,public,readonly,require,set,string,type,undefined,unknown},
}

\lstset{
    backgroundcolor=\color{backcolour},   
    commentstyle=\color{codegreen},
    keywordstyle=\color{magenta},
    numberstyle=\tiny\color{codegray},
    stringstyle=\color{codepurple},
    basicstyle=\ttfamily\footnotesize,
    breakatwhitespace=false,         
    breaklines=true,                 
    captionpos=b,                    
    keepspaces=true,                 
    numbers=left,                    
    numbersep=5pt,                  
    showspaces=false,                
    showstringspaces=false,
    showtabs=false,                  
    tabsize=2
}

% Định dạng tiêu đề chương theo chuẩn: CHƯƠNG 1 (Căn giữa, viết hoa)
\titleformat{\chapter}[display]
  {\normalfont\large\bfseries\centering}
  {\MakeUppercase{\chaptername}\ \thechapter}
  {5pt}
  {\large\MakeUppercase}
\titlespacing*{\chapter}{0pt}{-20pt}{30pt}

% Định dạng tiêu đề mục (Section, Subsection)
\titleformat{\section}{\normalfont\normalsize\bfseries}{\thesection}{1em}{}
\titleformat{\subsection}{\normalfont\normalsize\bfseries\itshape}{\thesubsection}{1em}{}

\usepackage{xurl}
\usepackage{hyperref}
\hypersetup{
    colorlinks=true,
    linkcolor=black,
    filecolor=magenta,      
    urlcolor=blue,
    pdftitle={Luận văn Tốt nghiệp - Smart LMS},
}

\begin{document}

% ======================================================
% TRANG BÌA
% ======================================================
\begin{titlepage}
    \centering
    {\large \textbf{BỘ GIÁO DỤC VÀ ĐÀO TẠO}} \\
    {\large \textbf{TRƯỜNG ĐẠI HỌC GIAO THÔNG VẬN TẢI TP.HCM}} \\
    {\large \textbf{KHOA CÔNG NGHỆ THÔNG TIN}}
    
    \vspace{1.5cm}
    \begin{figure}[h]
      \centering
      \IfFileExists{docs/luanvan/figures/uth-logo.png}{
        \includegraphics[width=0.35\textwidth]{docs/luanvan/figures/uth-logo.png}
      }{
        \IfFileExists{figures/uth-logo.png}{
          \includegraphics[width=0.35\textwidth]{figures/uth-logo.png}
        }{
          \framebox{\parbox{0.35\textwidth}{\centering
            \vspace{1cm}
            \textbf{LOGO TRƯỜNG} \\
            \small\textit{(Image Placeholder)}
            \vspace{1cm}
          }}
        }
      }
    \end{figure}
    
    \vspace{1.5cm}
    {\Large \textbf{LUẬN VĂN TỐT NGHIỆP ĐẠI HỌC}} \\
    
    \vspace{1cm}
    \begin{spacing}{1.5}
        {\huge \textbf{NGHIÊN CỨU VÀ XÂY DỰNG HỆ THỐNG QUẢN LÝ HỌC TẬP THÔNG MINH (SMART LMS) DỰA TRÊN KIẾN TRÚC RAG VÀ MÔ HÌNH NGÔN NGỮ LỚN}}
    \end{spacing}
    
    \vspace{2cm}
    \begin{flushright}
        \begin{tabular}{l p{6cm}}
            \textbf{Giảng viên hướng dẫn:} & \textbf{ThS. Bùi Trọng Hiếu} \\
            \textbf{Sinh viên thực hiện:}     & \textbf{Nguyễn Quốc Tùng (MS: 2251120259)} \\
                                              & \textbf{Nguyễn Võ Thành (MS: 2251120246)} \\
            \textbf{Lớp:}                     & \textbf{CN22E}
        \end{tabular}
    \end{flushright}

    \vfill
    {\large \textbf{TP. HỒ CHÍ MINH, NĂM 2025}}
\end{titlepage}

% ======================================================
% PHẦN TRƯỚC NỘI DUNG CHÍNH
% ======================================================
\pagenumbering{roman}

\chapter*{Lời cam đoan}
\phantomsection
\addcontentsline{toc}{chapter}{Lời cam đoan}
Chúng tôi xin cam đoan đây là công trình nghiên cứu của nhóm dưới sự hướng dẫn của giảng viên hướng dẫn. Các kết quả nêu trong luận văn là trung thực và chưa từng được công bố trong bất kỳ công trình nào khác. Mọi sự giúp đỡ cho việc thực hiện luận văn này đã được cảm ơn và các thông tin trích dẫn trong luận văn đã được chỉ rõ nguồn gốc.

\chapter*{Lời mở đầu}
\addcontentsline{toc}{chapter}{Lời mở đầu}
Sự bùng nổ của Trí tuệ nhân tạo tạo sinh (Generative AI) đã mở ra những cơ hội chưa từng có trong việc cá nhân hóa giáo dục. Tuy nhiên, các hệ thống Quản lý học tập (LMS) hiện nay tại Việt Nam chủ yếu vẫn đóng vai trò là kho lưu trữ tài liệu tĩnh, thiếu đi tính tương tác thông minh và khả năng hỗ trợ học sinh học tập chủ động.

Dự án này được thực hiện với mong muốn xây dựng một hệ sinh thái học tập "Smart LMS" dành cho cấp trung học. Điểm đột phá của hệ thống là việc ứng dụng kỹ thuật Retrieval-Augmented Generation (RAG) để tạo ra một Trợ lý ảo (AI Tutor) có khả năng hiểu sâu sắc nội dung bài giảng của giáo viên, từ đó phản hồi chính xác thắc mắc của học sinh. Bên cạnh đó, hệ thống tích hợp các công cụ giám sát thi cử dựa trên AI nhằm đảm bảo tính liêm chính trong môi trường học tập trực tuyến.

\newpage

\chapter*{Lời cảm ơn}
\phantomsection
\addcontentsline{toc}{chapter}{Lời cảm ơn}
Trong quá trình thực hiện luận văn và xây dựng hệ thống \textbf{EduVerse}, chúng tôi đã nhận được sự hướng dẫn, hỗ trợ và động viên từ nhiều cá nhân và đơn vị. Chúng tôi xin bày tỏ lòng biết ơn sâu sắc đến \textbf{ThS. Bùi Trọng Hiếu} -- người đã tận tình định hướng, góp ý chuyên môn và tạo điều kiện thuận lợi để nhóm hoàn thành đề tài. Những nhận xét khoa học, sự nghiêm khắc trong phương pháp làm việc và tinh thần trách nhiệm của thầy là nền tảng quan trọng giúp chúng tôi hình thành tư duy nghiên cứu và hoàn thiện sản phẩm theo đúng chuẩn mực học thuật.

    Chúng tôi xin trân trọng cảm ơn \textbf{Khoa Công nghệ Thông tin -- Trường Đại học Giao thông vận tải TP.HCM} đã cung cấp môi trường học tập, tài liệu tham khảo và cơ sở hạ tầng cần thiết trong suốt quá trình học tập cũng như khi triển khai luận văn. Đồng thời, chúng tôi xin cảm ơn các thầy cô trong Khoa đã trang bị những kiến thức nền tảng về công nghệ phần mềm, cơ sở dữ liệu và trí tuệ nhân tạo -- là những mảnh ghép không thể thiếu để hiện thực hóa đề tài.

Cuối cùng, chúng tôi xin cảm ơn gia đình và bạn bè đã luôn đồng hành, khích lệ tinh thần và tạo điều kiện về thời gian để có thể tập trung hoàn thành luận văn. Mặc dù đã cố gắng hết sức, luận văn khó tránh khỏi những thiếu sót; chúng tôi kính mong nhận được sự góp ý của quý thầy cô để tiếp tục hoàn thiện trong học tập và công việc sau này.

\newpage

\chapter*{Tóm tắt luận văn}
\phantomsection
\addcontentsline{toc}{chapter}{Tóm tắt luận văn}
Trong bối cảnh chuyển đổi số giáo dục và sự phát triển mạnh mẽ của Trí tuệ nhân tạo tạo sinh, các hệ thống Quản lý học tập (LMS) truyền thống tại Việt Nam vẫn chủ yếu đóng vai trò lưu trữ học liệu, thiếu khả năng tương tác thông minh và hạn chế trong việc giám sát liêm chính học thuật. Dự án nghiên cứu và xây dựng hệ thống \textbf{EduVerse} -- một nền tảng Smart LMS tích hợp mô-đun \textbf{AI Tutor} theo kiến trúc \textbf{Retrieval-Augmented Generation (RAG)} và mô-đun \textbf{chống gian lận} dựa trên nhật ký sự kiện[cite: 28, 29].

Về mặt kỹ thuật, hệ thống sử dụng \textbf{Next.js App Router}, \textbf{PostgreSQL} với \textbf{pgvector} để lưu trữ embedding và thực hiện truy vấn tương đồng[cite: 30, 31, 32]. Lớp AI được vận hành bởi \textbf{Gemini API} với mô hình \textbf{\texttt{gemini-2.5-flash-lite}} cho các tác vụ sinh nội dung và \texttt{gemini-embedding-001} cho tạo vector ngữ nghĩa[cite: 33, 34]. 

Kết quả đo thực nghiệm end-to-end cho thấy hiệu năng ấn tượng của hệ thống: độ trễ phản hồi trung bình của API RAG Tutor đạt \textbf{5.2s (P95 5.5s)} [cite: 320]; cơ chế ghi nhận \texttt{ExamEvent} đạt \textbf{200/200} yêu cầu thành công với độ trễ trung bình cực thấp \textbf{0.8s (P95 1.4s)} [cite: 321, 323]; và mô-đun AI Anti-cheat Summary đạt độ trễ trung bình \textbf{5.6s (P95 6.3s)}[cite: 322]. Những kết quả này minh chứng cho tính khả thi và hiệu quả của việc ứng dụng LLM thế hệ mới vào môi trường giáo dục trực tuyến[cite: 352, 353].

\newpage
\tableofcontents 
\newpage

\chapter*{Danh mục từ viết tắt}
\phantomsection
\addcontentsline{toc}{chapter}{Danh mục từ viết tắt}
\begin{table}[H]
\centering
\small
\begin{tabularx}{\textwidth}{|p{3cm}|X|}
\hline
\textbf{Từ viết tắt} & \textbf{Giải thích} \\ \hline
LMS & Learning Management System -- Hệ thống quản lý học tập. \\ \hline
RAG & Retrieval-Augmented Generation -- Sinh nội dung tăng cường truy xuất tri thức từ kho tài liệu. \\ \hline
LLM & Large Language Model -- Mô hình ngôn ngữ lớn. \\ \hline
RBAC & Role-Based Access Control -- Kiểm soát truy cập theo vai trò người dùng. \\ \hline
JWT & JSON Web Token -- Token dạng JSON dùng để xác thực/ủy quyền. \\ \hline
ORM & Object-Relational Mapping -- Ánh xạ đối tượng--quan hệ. \\ \hline
API & Application Programming Interface -- Giao diện lập trình ứng dụng. \\ \hline
ACID & Atomicity, Consistency, Isolation, Durability -- Thuộc tính đảm bảo tính đúng đắn của giao dịch CSDL. \\ \hline
UI/UX & User Interface/User Experience -- Giao diện người dùng/Trải nghiệm người dùng. \\ \hline
\end{tabularx}
\end{table}

\newpage
\listoffigures
\newpage
\listoftables

% ======================================================
% CHƯƠNG 1: MỞ ĐẦU
% ======================================================
\newpage
\pagenumbering{arabic}

\chapter{MỞ ĐẦU}

\section{Tính cấp thiết của đề tài}
Trong kỷ nguyên số hóa giáo dục, các hệ thống Quản lý học tập (LMS - Learning Management System) đóng vai trò then chốt trong việc kết nối người dạy và người học. Tuy nhiên, các nền tảng truyền thống hiện nay tại Việt Nam vẫn tồn tại những hạn chế đáng kể:
\begin{itemize}
    \item \textbf{Thiếu tính tương tác cá nhân hóa:} Học liệu chủ yếu ở dạng tĩnh (PDF, video), người học khó tìm được sự trợ giúp tức thời khi gặp vướng mắc về kiến thức trong bài giảng.
    \item \textbf{Thách thức về liêm chính học thuật:} Việc đánh giá trực tuyến đối mặt với rủi ro gian lận ngày càng tinh vi. Các giải pháp hiện tại thường chỉ dừng lại ở việc khóa trình duyệt cơ bản, thiếu khả năng phân tích hành vi sâu sắc.
    \item \textbf{Hạn chế về hạ tầng lưu trữ và mở rộng:} Việc quản lý khối lượng lớn tài liệu đa phương tiện và dữ liệu vector đòi hỏi một hạ tầng đám mây mạnh mẽ, điều mà nhiều hệ thống tự triển khai (on-premise) khó đáp ứng hiệu quả.
\end{itemize}

Sự bùng nổ của Trí tuệ nhân tạo tạo sinh (Generative AI) và các hạ tầng đám mây hiện đại đã mở ra cơ hội mới. Việc tích hợp AI cùng các dịch vụ Cloud-native không chỉ giúp hệ thống thông minh hơn mà còn đảm bảo tính sẵn sàng cao. Xuất phát từ thực tiễn đó, nhóm thực hiện đề tài: \textbf{"Nghiên cứu và xây dựng hệ thống quản lý học tập thông minh (Smart LMS) dựa trên kiến trúc RAG và mô hình ngôn ngữ lớn"} trên nền tảng hạ tầng hiện đại.

\section{Mục tiêu nghiên cứu}
Mục tiêu chính của đề tài là thiết kế và triển khai hệ thống Smart LMS mang tên \textbf{EduVerse}, cụ thể:
\begin{itemize}
    \item Nghiên cứu kiến trúc \textbf{Retrieval-Augmented Generation (RAG)} để xây dựng tính năng AI Tutor, cho phép phản hồi thông minh dựa trên chính dữ liệu bài giảng của giáo viên.
    \item Triển khai hạ tầng lưu trữ đám mây tích hợp để quản lý tập trung tài liệu học tập và dữ liệu người dùng một cách bảo mật và tối ưu.
    \item Xây dựng cơ chế giám sát thi cử thông minh dựa trên nhật ký sự kiện (\textbf{ExamEvent Logging}), sử dụng AI để phân tích và cảnh báo hành vi gian lận.
    \item Tối ưu hóa hiệu năng hệ thống để đảm bảo độ trễ thấp (Latency) khi xử lý các tác vụ AI phức tạp.
\end{itemize}

\section{Đối tượng và phạm vi nghiên cứu}
\begin{itemize}
    \item \textbf{Đối tượng nghiên cứu:} Các mô hình ngôn ngữ lớn (LLM), trọng tâm là mô hình \textbf{Gemini 2.5 Flash Lite}; kiến trúc hệ thống RAG; cơ sở dữ liệu vector và các dịch vụ hạ tầng đám mây.
    \item \textbf{Phạm vi công nghệ:} Hệ thống được phát triển trên nền tảng \textbf{Next.js 14} (App Router), sử dụng \textbf{Prisma ORM} kết hợp với hệ sinh thái \textbf{Supabase} (bao gồm PostgreSQL cho lưu trữ dữ liệu, \textbf{pgvector} cho dữ liệu không gian vector và \textbf{Supabase Storage} cho quản lý tệp tin).
    \item \textbf{Phạm vi ứng dụng:} Áp dụng trong môi trường đào tạo trực tuyến tại các trường phổ thông và đại học.
\end{itemize}

\section{Phương pháp nghiên cứu}
Đề tài sử dụng kết hợp các phương pháp:
\begin{itemize}
    \item \textbf{Phương pháp nghiên cứu lý thuyết:} Tìm hiểu về cơ chế hoạt động của LLM, kỹ thuật Prompt Engineering và quy trình xử lý RAG.
    \item \textbf{Phương pháp thực nghiệm phần mềm:} Xây dựng sản phẩm theo mô hình Agile, triển khai trên môi trường \textbf{Vercel} và \textbf{Supabase Cloud} để đo lường các chỉ số hiệu năng thực tế.
    \item \textbf{Phương pháp thống kê:} Thu thập và phân tích dữ liệu thực nghiệm về độ trễ API và tính ổn định của hạ tầng đám mây.
\end{itemize}

\section{Ý nghĩa khoa học và thực tiễn}
\begin{itemize}
    \item \textbf{Về mặt khoa học:} Đóng góp một mô hình triển khai Smart LMS thực tiễn kết hợp giữa AI tạo sinh và hạ tầng đám mây linh hoạt.
    \item \textbf{Về mặt thực tiễn:} Giải quyết bài toán quá tải trong hỗ trợ học tập và nâng cao sự minh bạch, liêm chính trong kiểm tra đánh giá trực tuyến.
\end{itemize}

\section{Cấu trúc luận văn}
Luận văn được cấu trúc thành 5 chương chính:
\begin{itemize}
    \item \textbf{Chương 1 - Mở đầu:} Trình bày tính cấp thiết, mục tiêu, phạm vi và công nghệ chủ chốt.
    \item \textbf{Chương 2 - Cơ sở lý thuyết:} Nghiên cứu về AI tạo sinh, kiến trúc RAG và các dịch vụ hạ tầng (Supabase, Next.js).
    \item \textbf{Chương 3 - Phân tích và Thiết kế hệ thống:} Mô tả các biểu đồ Use Case, Sequence và thiết kế CSDL trên PostgreSQL.
    \item \textbf{Chương 4 - Triển khai hệ thống:} Chi tiết các module AI Tutor, Anti-cheat và quy trình xử lý tệp tin trên Supabase Storage.
    \item \textbf{Chương 5 - Thực nghiệm và Đánh giá:} Trình bày kết quả đo lường hiệu năng và đánh giá tổng thể.
\end{itemize}

% ======================================================
% CHƯƠNG 2: CƠ SỞ LÝ THUYẾT VÀ CÔNG NGHỆ
% ======================================================

\chapter{CƠ SỞ LÝ THUYẾT VÀ CÔNG NGHỆ}

\section{Trí tuệ nhân tạo tạo sinh (Generative AI)}
Trí tuệ nhân tạo tạo sinh (Generative AI) là một nhánh của học máy có khả năng tạo ra nội dung mới như văn bản, hình ảnh, mã nguồn và âm thanh dựa trên dữ liệu đã được huấn luyện. Sự dịch chuyển từ AI phân tích (Discriminative AI) sang AI tạo sinh đã mở ra kỷ nguyên mới cho giáo dục số, nơi AI không chỉ phân loại kết quả mà còn có khả năng kiến tạo tri thức và hỗ trợ tương tác.

\section{Mô hình ngôn ngữ lớn và Gemini 2.5 Flash Lite}
\subsection{Tổng quan về Mô hình ngôn ngữ lớn (LLM)}
Mô hình ngôn ngữ lớn (Large Language Model - LLM) là các mô hình học sâu với hàng tỷ tham số, được huấn luyện trên tập dữ liệu khổng lồ để hiểu và sinh ngôn ngữ tự nhiên. LLM hoạt động dựa trên cơ chế Attention (Chú ý) giúp mô hình hiểu được ngữ cảnh phức tạp và mối liên hệ giữa các từ trong văn bản.

\subsection{Mô hình Gemini 2.5 Flash Lite}
Hệ thống \textbf{EduVerse} sử dụng mô hình \textbf{Gemini 2.5 Flash Lite} làm hạt nhân xử lý. Đây là phiên bản tối ưu của Google hướng đến sự cân bằng giữa hiệu suất và chi phí:
\begin{itemize}
    \item \textbf{Tốc độ phản hồi (Latency):} Được tối ưu hóa cho các tác vụ thời gian thực, phù hợp cho trợ giảng ảo (AI Tutor) cần phản hồi tức thì.
    \item \textbf{Khả năng đa phương thức (Multimodal):} Hiểu văn bản, hình ảnh và video bài giảng một cách đồng nhất.
    \item \textbf{Cửa sổ ngữ cảnh (Context Window):} Hỗ trợ lượng token lớn, cho phép đưa toàn bộ nội dung một khóa học vào ngữ cảnh để AI xử lý mà không bị mất thông tin.
    \item \textbf{JSON Mode:} Khả năng xuất dữ liệu dưới định dạng cấu trúc, giúp hệ thống tích hợp trực tiếp kết quả AI vào logic xử lý của backend.
\end{itemize}

\section{Kiến trúc Retrieval-Augmented Generation (RAG)}
Kiến trúc RAG là giải pháp đột phá giúp LLM khắc phục hiện tượng "ảo tưởng" (hallucination) bằng cách kết hợp sức mạnh sáng tạo của AI với dữ liệu tin cậy từ bên ngoài.

\subsection{Quy trình hoạt động của RAG}
Quy trình triển khai RAG trong hệ thống bao gồm các bước:
\begin{enumerate}
    \item \textbf{Lập chỉ mục (Indexing):} Văn bản bài giảng được chia nhỏ (Chunking), chuyển đổi thành vector thông qua mô hình Embedding (\texttt{gemini-embedding-001}) và lưu vào Vector Database.
    \item \textbf{Truy xuất (Retrieval):} Khi người dùng đặt câu hỏi, hệ thống chuyển câu hỏi đó thành vector và thực hiện tìm kiếm tương đồng ngữ nghĩa.
    \item \textbf{Tăng cường (Augmentation):} Các đoạn văn bản có độ liên quan cao nhất được trích xuất và đưa vào câu lệnh (Prompt) gửi đến LLM.
    \item \textbf{Tạo nội dung (Generation):} LLM dựa trên ngữ cảnh được cung cấp để đưa ra câu trả lời chính xác và có trích dẫn từ bài học.
\end{enumerate}

\subsection{Kỹ thuật chia nhỏ văn bản (Text Chunking)}
Trong hệ thống, kỹ thuật Chunking đóng vai trò then chốt. Việc chia nhỏ văn bản giúp:
\begin{itemize}
    \item Giữ cho kích thước dữ liệu đầu vào nằm trong giới hạn của mô hình.
    \item Tăng độ chính xác khi truy xuất (Retrieval Precision) vì mỗi đoạn văn bản sẽ tập trung vào một ý chính duy nhất.
\end{itemize}

\section{Cơ sở dữ liệu Vector và pgvector}
Để lưu trữ và truy vấn dữ liệu embedding, hệ thống sử dụng \textbf{pgvector} -- một tiện ích mở rộng cho hệ quản trị cơ sở dữ liệu PostgreSQL.

\subsection{Độ tương đồng Cosine (Cosine Similarity)}
Hệ thống thực hiện so sánh vector dựa trên công thức độ tương đồng Cosine:
\begin{equation}
    \text{similarity} = \cos(\theta) = \frac{\mathbf{A} \cdot \mathbf{B}}{\|\mathbf{A}\| \|\mathbf{B}\|}
\end{equation}
Trong đó $\mathbf{A}$ và $\mathbf{B}$ là vector embedding của câu hỏi và nội dung bài giảng. Giá trị càng gần 1 thể hiện mức độ tương đồng về ngữ nghĩa càng cao.

\subsection{Lợi thế của pgvector tích hợp}
Khác với các cơ sở dữ liệu vector độc lập, việc sử dụng pgvector giúp hệ thống quản lý cả dữ liệu quan hệ (Thông tin sinh viên, điểm số) và dữ liệu không gian vector trên cùng một hạ tầng duy nhất, đảm bảo tính nhất quán và hiệu năng truy vấn.

\section{Nền tảng hạ tầng và Công nghệ phát triển}
\subsection{Framework Next.js 14}
Next.js 14 với kiến trúc \textbf{App Router} cung cấp các tính năng hiện đại:
\begin{itemize}
    \item \textbf{React Server Components (RSC):} Tối ưu hóa tốc độ tải trang bằng cách thực hiện các truy vấn dữ liệu nặng ở phía Server.
    \item \textbf{Server Actions:} Đơn giản hóa việc tương tác giữa Client và Database mà không cần thông qua các API endpoint trung gian phức tạp.
\end{itemize}

\subsection{Hệ sinh thái Supabase}
Supabase đóng vai trò là nền tảng Backend-as-a-Service (BaaS) cung cấp:
\begin{itemize}
    \item \textbf{PostgreSQL Database:} Lưu trữ toàn bộ dữ liệu nghiệp vụ của hệ thống LMS.
    \item \textbf{Supabase Storage:} Lưu trữ các tệp tin bài giảng (PDF, hình ảnh) và bài nộp của sinh viên với cơ chế bảo mật cao.
\end{itemize}

\subsection{Prisma ORM}
Prisma được sử dụng như một lớp trừu tượng hóa cơ sở dữ liệu (ORM), giúp quản lý Schema một cách Type-safe (an toàn về kiểu dữ liệu) và tối ưu hóa các câu lệnh truy vấn SQL phức tạp cho các báo cáo học tập.

\section{Lý thuyết về Giám sát hành vi và Liêm chính học thuật}
Tính năng AI Anti-cheat dựa trên lý thuyết \textbf{Event-driven Monitoring} (Giám sát dựa trên sự kiện). Hệ thống thu thập các hành vi (ExamEvents) như: mất tập trung (Tab blur), thay đổi clipboard, hoặc thoát khỏi chế độ toàn màn hình. 
Dữ liệu này được tổng hợp thành một "Điểm nghi ngờ" (Suspicious Score) thông qua các thuật toán thống kê, giúp AI đưa ra nhận định khách quan về tính trung thực của thí sinh thay vì chỉ sử dụng các biện pháp ngăn chặn cứng nhắc.

% ======================================================
% CHƯƠNG 3: PHÂN TÍCH VÀ THIẾT KẾ HỆ THỐNG
% ======================================================

% ======================================================
% CHƯƠNG 3: PHÂN TÍCH VÀ THIẾT KẾ HỆ THỐNG
% ======================================================

\chapter{PHÂN TÍCH VÀ THIẾT KẾ HỆ THỐNG}

\section{Phân tích yêu cầu hệ thống}

\subsection{Xác định các tác nhân (Actors) và phân quyền (RBAC)}
Hệ thống \textbf{EduVerse} phục vụ 4 nhóm người dùng chính với quyền hạn tách biệt theo mô hình \textbf{Role-Based Access Control (RBAC)}. Vai trò được lưu trữ tại bảng \texttt{users} (thuộc tính \texttt{role} trong Prisma model \texttt{User}) và được kiểm soát tại tầng middleware/API thông qua cơ chế xác thực và kiểm tra quyền truy cập.

\begin{itemize}
  \item \textbf{Quản trị viên (ADMIN):} Quản lý người dùng, lớp học, tổ chức, cấu hình hệ thống, theo dõi nhật ký (audit logs).
  \item \textbf{Giáo viên (TEACHER):} Tạo lớp học, tạo khóa học/bài giảng, tạo bài tập (Quiz/Essay), giao bài cho lớp, theo dõi bài nộp, chấm điểm, giám sát chống gian lận.
  \item \textbf{Học sinh (STUDENT):} Tham gia lớp, học bài giảng, làm bài tập, nộp bài, gửi sự kiện chống gian lận trong lúc làm bài, nhận thông báo, chat.
  \item \textbf{Phụ huynh (PARENT):} Liên kết với học sinh, xem tiến độ/điểm, nhận tóm tắt tuần, chat với giáo viên theo cơ chế kiểm soát quyền.
\end{itemize}

\subsection{Yêu cầu chức năng (Functional Requirements)}
Các nhóm chức năng chính của hệ thống bao gồm:
\begin{itemize}
  \item \textbf{Xác thực và quản trị người dùng:} đăng ký/đăng nhập, quên mật khẩu, quản lý tài khoản theo RBAC.
  \item \textbf{Quản lý lớp học và học liệu:} lớp học (classroom), khóa học (course), bài giảng (lesson), tệp đính kèm bài giảng.
  \item \textbf{Quản lý bài tập và đánh giá:} tạo bài Essay/Quiz, attempt cho Quiz, nộp bài, chấm tự động Quiz, chấm thủ công, quản lý tệp đính kèm bài tập.
  \item \textbf{AI Tutor theo kiến trúc RAG:} lập chỉ mục embedding cho bài giảng, truy xuất ngữ nghĩa bằng pgvector, sinh câu trả lời bằng Gemini.
  \item \textbf{Giám sát thi và chống gian lận:} ghi nhận \texttt{ExamEvent}, tính điểm nghi ngờ (suspicion score), sinh tóm tắt AI phục vụ giám sát.
  \item \textbf{Giao tiếp:} announcements lớp học (bài đăng, bình luận, tệp đính kèm), chat (tin nhắn, tệp đính kèm), notifications (thông báo hệ thống).
  \item \textbf{Phụ huynh:} liên kết Parent--Student, sinh tóm tắt tuần (cron) và gửi thông báo.
\end{itemize}

\subsection{Yêu cầu phi chức năng (Non-functional Requirements)}
\begin{itemize}
  \item \textbf{Hiệu năng:} các API nghiệp vụ thường xuyên (announcements, submissions, notifications, exam-events) có độ trễ thấp; các API AI (RAG Tutor, Anti-cheat summary, AI grading) chấp nhận độ trễ cao hơn do phụ thuộc mô hình LLM.
  \item \textbf{Tính nhất quán dữ liệu:} đảm bảo ACID cho các thao tác ghi điểm/nộp bài, tránh ghi trùng attempt, hỗ trợ cơ chế dedupe thông báo (trường \texttt{dedupeKey}).
  \item \textbf{Bảo mật:} mật khẩu được hash; RBAC bắt buộc; dữ liệu tệp lưu trên Supabase Storage và truy cập qua signed URL; API có validation (Zod) và rate limit ở các route nhạy cảm.
  \item \textbf{Khả năng mở rộng:} sử dụng PostgreSQL + pgvector để lưu embedding, hỗ trợ tăng số lượng bài giảng/embedding mà vẫn đảm bảo truy xuất nhanh theo khóa học.
\end{itemize}

\section{Thiết kế kiến trúc hệ thống}

\subsection{Kiến trúc tổng thể}
EduVerse áp dụng kiến trúc \textbf{Modern Full-stack} với Next.js App Router làm lõi, bao gồm:
\begin{itemize}
  \item \textbf{Presentation Layer:} giao diện Web (RSC/Client Components).
  \item \textbf{Application Layer:} Next.js API Routes, xác thực (NextAuth), logic nghiệp vụ, gọi dịch vụ AI.
  \item \textbf{Data Layer:} PostgreSQL (Prisma ORM), pgvector (embedding), Supabase Storage (lưu tệp).
  \item \textbf{AI Layer:} Gemini API (tạo embedding và sinh nội dung).
\end{itemize}

\begin{figure}[H]
  \centering
  \includegraphics[width=0.95\textwidth]{figures/uml/EduVerse_Component_Architecture.png}
  \caption{Sơ đồ kiến trúc thành phần hệ thống EduVerse}
  \label{fig:component-architecture}
\end{figure}

\subsection{Nguyên tắc thiết kế API}
Các API được thiết kế theo hướng:
\begin{itemize}
  \item \textbf{Xác thực/ủy quyền nhất quán:} kiểm tra người dùng và role trước khi truy cập tài nguyên.
  \item \textbf{Validation đầu vào:} sử dụng Zod (các schema) để kiểm tra query/body nhằm giảm lỗi dữ liệu.
  \item \textbf{Chuẩn hóa lỗi:} trả về cấu trúc lỗi thống nhất (\texttt{errorResponse}) để frontend dễ xử lý.
  \item \textbf{Tệp và Storage:} upload xử lý server-side; download/preview thông qua signed URL có TTL.
\end{itemize}

\section{Phân tích Use Case}

\subsection{Use Case tổng quát}
Sơ đồ Use Case tổng quát thể hiện nhóm chức năng và phân quyền theo tác nhân.
\begin{figure}[H]
  \centering
  \includegraphics[width=0.95\textwidth]{figures/uml/usecase-overall.png}
  \caption{Sơ đồ Use Case tổng quát của hệ thống}
  \label{fig:usecase-overall}
\end{figure}

\subsection{Use Case chi tiết cho Học sinh (Student)}
\begin{figure}[H]
  \centering
  \includegraphics[width=0.95\textwidth]{figures/uml/usecase-student.png}
  \caption{Sơ đồ Use Case chi tiết cho tác nhân Học sinh}
  \label{fig:usecase-student}
\end{figure}

\subsection{Use Case chi tiết cho Giáo viên (Teacher)}
\begin{figure}[H]
  \centering
  \includegraphics[width=0.95\textwidth]{figures/uml/usecase-teacher-extended.png}
  \caption{Sơ đồ Use Case chi tiết (mở rộng) cho tác nhân Giáo viên}
  \label{fig:usecase-teacher-extended}
\end{figure}

\subsection{Use Case chi tiết cho Phụ huynh (Parent)}
\begin{figure}[H]
  \centering
  \includegraphics[width=0.95\textwidth]{figures/uml/usecase-parent.png}
  \caption{Sơ đồ Use Case chi tiết cho tác nhân Phụ huynh}
  \label{fig:usecase-parent}
\end{figure}

\subsection{Use Case chi tiết cho Quản trị viên (Admin)}
\begin{figure}[H]
  \centering
  \includegraphics[width=0.95\textwidth]{figures/uml/usecase-admin.png}
  \caption{Sơ đồ Use Case chi tiết cho tác nhân Quản trị viên}
  \label{fig:usecase-admin}
\end{figure}

\section{Thiết kế quy trình nghiệp vụ (Sequence / Activity / State)}

\subsection{Quy trình RAG cho AI Tutor}
Quy trình RAG đảm bảo câu trả lời AI dựa trên tri thức từ bài giảng (lesson) đã được lập chỉ mục embedding:
\begin{itemize}
  \item Lập chỉ mục: tách nội dung bài giảng thành các đoạn (chunks), tạo embedding và lưu vào bảng \texttt{lesson\_embedding\_chunks} (model \texttt{LessonEmbeddingChunk}).
  \item Truy xuất: khi học sinh hỏi, hệ thống embed câu hỏi và truy vấn tương đồng (pgvector) để lấy top-k chunks.
  \item Sinh câu trả lời: ghép chunks vào prompt và gọi Gemini để sinh nội dung trả lời.
\end{itemize}

\begin{figure}[H]
  \centering
  \includegraphics[width=0.92\textwidth]{figures/uml/flow-rag.png}
  \caption{Sơ đồ luồng xử lý RAG cho AI Tutor}
  \label{fig:flow-rag}
\end{figure}

\subsection{Quy trình ghi nhận và xử lý sự kiện chống gian lận (Anti-cheat)}
Khi học sinh làm bài, frontend ghi nhận hành vi bất thường và gửi về backend dưới dạng \texttt{ExamEvent}. Backend lưu trữ, cảnh báo, và cho phép giáo viên tổng hợp điểm nghi ngờ.

\begin{figure}[H]
  \centering
  \includegraphics[width=0.95\textwidth]{figures/uml/seq-exam-events.png}
  \caption{Sơ đồ tuần tự ghi nhận sự kiện thi cử (Exam Events)}
  \label{fig:seq-exam-events}
\end{figure}

Để thống nhất dữ liệu, hệ thống chuẩn hóa/kiểm soát sự kiện trước khi lưu và trước khi tính điểm.
\begin{figure}[H]
  \centering
  \includegraphics[width=0.95\textwidth]{figures/uml/activity-anti-cheat-normalize.png}
  \caption{Sơ đồ hoạt động xử lý pipeline Anti-cheat (Normalize--Store--Notify)}
  \label{fig:activity-anti-cheat-normalize}
\end{figure}

Giáo viên có thể yêu cầu AI sinh tóm tắt anti-cheat dựa trên dữ liệu \texttt{exam\_events} và điểm nghi ngờ.
\begin{figure}[H]
  \centering
  \includegraphics[width=0.98\textwidth]{figures/uml/seq-anti-cheat-ai-summary.png}
  \caption{Sơ đồ tuần tự yêu cầu AI tóm tắt Anti-cheat}
  \label{fig:seq-anti-cheat-ai-summary}
\end{figure}

\subsection{Quy trình làm bài Quiz: tạo attempt và nộp bài}
Bài Quiz hỗ trợ attempt, shuffle seed, và tự động chấm điểm:
\begin{figure}[H]
  \centering
  \includegraphics[width=0.98\textwidth]{figures/uml/seq-quiz-attempt-submit.png}
  \caption{Sơ đồ tuần tự quy trình bắt đầu attempt và nộp bài Quiz}
  \label{fig:seq-quiz-attempt-submit}
\end{figure}

Trạng thái attempt được quản lý nhằm hỗ trợ giám sát/can thiệp:
\begin{figure}[H]
  \centering
  \includegraphics[width=0.75\textwidth]{figures/uml/state-assignment-attempt.png}
  \caption{Sơ đồ trạng thái của \texttt{AssignmentAttempt} (Quiz)}
  \label{fig:state-assignment-attempt}
\end{figure}

\subsection{Quy trình nộp bài dạng tệp (file-based submission)}
Hệ thống hỗ trợ nộp bài dạng tệp với mô hình:
\textbf{server-side upload} (API upload file lên Supabase Storage) + lưu metadata vào DB + xác nhận nộp.

\begin{figure}[H]
  \centering
  \includegraphics[width=0.98\textwidth]{figures/uml/seq-file-submission-server-upload.png}
  \caption{Sơ đồ tuần tự nộp bài dạng tệp (server-side upload)}
  \label{fig:seq-file-submission-server-upload}
\end{figure}

Trạng thái bài nộp dạng tệp:
\begin{figure}[H]
  \centering
  \includegraphics[width=0.6\textwidth]{figures/uml/state-file-submission.png}
  \caption{Sơ đồ trạng thái của \texttt{Submission} (file-based)}
  \label{fig:state-file-submission}
\end{figure}

\subsection{Quy trình quản lý bài đăng (Announcements) trong lớp}
Giáo viên đăng bài cho lớp và hệ thống gửi thông báo cho học sinh trong lớp:
\begin{figure}[H]
  \centering
  \includegraphics[width=0.98\textwidth]{figures/uml/seq-announcement-post-and-notify.png}
  \caption{Sơ đồ tuần tự đăng announcement và notify học sinh}
  \label{fig:seq-announcement-post-and-notify}
\end{figure}

Bình luận và trả lời bình luận có cơ chế thông báo (notify teacher, notify student khi được reply):
\begin{figure}[H]
  \centering
  \includegraphics[width=0.98\textwidth]{figures/uml/seq-announcement-comments-and-replies.png}
  \caption{Sơ đồ tuần tự bình luận/trả lời trong announcement}
  \label{fig:seq-announcement-comments-and-replies}
\end{figure}

Tệp đính kèm announcement được upload server-side và tải xuống qua signed URL:
\begin{figure}[H]
  \centering
  \includegraphics[width=0.98\textwidth]{figures/uml/seq-announcement-attachments.png}
  \caption{Sơ đồ tuần tự upload/download tệp đính kèm announcement}
  \label{fig:seq-announcement-attachments}
\end{figure}

\subsection{Quy trình quản lý tệp đính kèm bài tập (Assignment Attachments)}
Giáo viên upload attachment cho bài tập và hệ thống tạo signed URL để giáo viên/học sinh tải về:
\begin{figure}[H]
  \centering
  \includegraphics[width=0.98\textwidth]{figures/uml/seq-assignment-attachments.png}
  \caption{Sơ đồ tuần tự upload và lấy signed URL cho tệp đính kèm bài tập}
  \label{fig:seq-assignment-attachments}
\end{figure}

\subsection{Quy trình Chat và tệp đính kèm}
Chat hỗ trợ gửi tin nhắn và upload tệp đính kèm; khi giáo viên nhắn tin phụ huynh sẽ nhận notification:
\begin{figure}[H]
  \centering
  \includegraphics[width=0.98\textwidth]{figures/uml/seq-chat-attachments.png}
  \caption{Sơ đồ tuần tự chat kèm tệp đính kèm}
  \label{fig:seq-chat-attachments}
\end{figure}

\subsection{Quy trình Notifications}
Người dùng xem danh sách thông báo và quản lý trạng thái đã đọc:
\begin{figure}[H]
  \centering
  \includegraphics[width=0.85\textwidth]{figures/uml/seq-notifications.png}
  \caption{Sơ đồ tuần tự quản lý Notifications}
  \label{fig:seq-notifications}
\end{figure}

\subsection{Quy trình liên kết Phụ huynh--Học sinh và tóm tắt tuần}
Hệ thống hỗ trợ:
\begin{itemize}
  \item phụ huynh gửi yêu cầu liên kết;
  \item học sinh duyệt yêu cầu;
  \item cron job sinh tóm tắt tuần và gửi notification cho phụ huynh.
\end{itemize}

\begin{figure}[H]
  \centering
  \includegraphics[width=0.98\textwidth]{figures/uml/seq-parent-link-and-weekly-summary.png}
  \caption{Sơ đồ tuần tự liên kết phụ huynh--học sinh và tóm tắt tuần}
  \label{fig:seq-parent-link-and-weekly-summary}
\end{figure}

\subsection{Quy trình khôi phục mật khẩu}
Người dùng khôi phục mật khẩu theo quy trình send-code/verify/reset:
\begin{figure}[H]
  \centering
  \includegraphics[width=0.98\textwidth]{figures/uml/seq-reset-password.png}
  \caption{Sơ đồ tuần tự khôi phục mật khẩu}
  \label{fig:seq-reset-password}
\end{figure}

\subsection{Quy trình Cron Job lập chỉ mục Embeddings bài giảng}
Cron job chạy định kỳ, có xác thực \texttt{CRON\_SECRET}, hỗ trợ skip lesson không đổi và ghi audit:
\begin{figure}[H]
  \centering
  \includegraphics[width=0.98\textwidth]{figures/uml/seq-cron-index-lesson-embeddings.png}
  \caption{Sơ đồ tuần tự Cron Job lập chỉ mục embeddings bài giảng}
  \label{fig:seq-cron-index-lesson-embeddings}
\end{figure}

\section{Thiết kế cơ sở dữ liệu (Database Design)}

\subsection{Tổng quan thiết kế dữ liệu}
Hệ thống sử dụng PostgreSQL làm CSDL chính và mở rộng pgvector để lưu embedding. Thiết kế dữ liệu kết hợp:
\begin{itemize}
  \item \textbf{Dữ liệu quan hệ}: users, classrooms, assignments, submissions, announcements, chat...
  \item \textbf{Dữ liệu bán cấu trúc}: các trường JSON như \texttt{anti\_cheat\_config}, \texttt{presentation}, \texttt{contentSnapshot}, \texttt{metadata}.
  \item \textbf{Dữ liệu vector}: embedding của lesson chunks trong bảng \texttt{lesson\_embedding\_chunks}.
\end{itemize}

\subsection{ERD cốt lõi}
\begin{figure}[H]
  \centering
  \includegraphics[width=1.0\textwidth]{figures/uml/EduVerse_DB_Core.png}
  \caption{Sơ đồ ERD cốt lõi của hệ thống EduVerse}
  \label{fig:erd-core}
\end{figure}

\subsection{ERD mở rộng: Auth / Admin / Audit}
\begin{figure}[H]
  \centering
  \includegraphics[width=1.0\textwidth]{figures/uml/db-auth-admin-audit.png}
  \caption{ERD mở rộng: phân hệ Xác thực, Quản trị và Nhật ký}
  \label{fig:db-auth-admin-audit}
\end{figure}

\subsection{ERD mở rộng: Learning \& Assessment}
\begin{figure}[H]
  \centering
  \includegraphics[width=1.0\textwidth]{figures/uml/db-learning-assessment.png}
  \caption{ERD mở rộng: phân hệ Học tập và Đánh giá}
  \label{fig:db-learning-assessment}
\end{figure}

\subsection{ERD mở rộng: Communication}
\begin{figure}[H]
  \centering
  \includegraphics[width=1.0\textwidth]{figures/uml/db-communication.png}
  \caption{ERD mở rộng: phân hệ Giao tiếp}
  \label{fig:db-communication}
\end{figure}

\section{Bảng Mapping Sơ đồ \texorpdfstring{$\leftrightarrow$}{} API \texorpdfstring{$\leftrightarrow$}{} Prisma Models}
Bảng mapping giúp đảm bảo tính nhất quán giữa tài liệu thiết kế và mã nguồn triển khai.

\begin{table}[H]
\centering
\small
\begin{tabularx}{\textwidth}{|p{4cm}|p{5.2cm}|p{4.2cm}|X|}
\hline
\textbf{Sơ đồ} & \textbf{API/Route tiêu biểu} & \textbf{Prisma models tiêu biểu} & \textbf{Mục đích} \\ \hline

\texttt{usecase-student} &
\texttt{/api/students/*}, \texttt{/api/submissions*}, \texttt{/api/exam-events}, \texttt{/api/ai/tutor/chat}, \texttt{/api/chat/*}, \texttt{/api/notifications} &
\texttt{User, ClassroomStudent, Assignment, Question, Option, AssignmentAttempt, AssignmentSubmission, Submission, SubmissionFile, ExamEvent, Notification, Conversation, Message, ChatAttachment} &
Chức năng học, làm bài, nộp bài, anti-cheat, chat/notification cho học sinh. \\ \hline

\texttt{usecase-teacher-extended} &
\texttt{/api/teachers/*}, \texttt{/api/assignments/*}, \texttt{/api/classrooms/*/announcements}, \texttt{/api/ai/anti-cheat/summary} &
\texttt{User, Classroom, Course, Lesson, LessonEmbeddingChunk, Assignment, AssignmentClassroom, AssignmentFile, ExamEvent, Notification, Announcement, AnnouncementComment, AnnouncementAttachment} &
Chức năng tạo nội dung, giao bài, giám sát/chấm bài, quản lý announcements. \\ \hline

\texttt{usecase-parent} &
\texttt{/api/parent/*}, \texttt{/api/cron/parent-weekly-summary}, \texttt{/api/chat/*}, \texttt{/api/notifications} &
\texttt{ParentStudent, ParentStudentLinkRequest, ParentStudentInvitation, Notification, Conversation, Message} &
Chức năng liên kết và theo dõi con, nhận tóm tắt tuần. \\ \hline

\texttt{usecase-admin} &
\texttt{/api/admin/*} &
\texttt{User, Organization, OrganizationMember, SystemSetting, AuditLog} &
Quản trị hệ thống và người dùng. \\ \hline

\texttt{seq-file-submission-server-upload} &
\texttt{POST /api/submissions/upload}, \texttt{POST /api/submissions}, \texttt{PUT /api/submissions}, \texttt{GET /api/submissions/signed-url} &
\texttt{Submission, SubmissionFile, Assignment, AssignmentClassroom, Notification} &
Luồng nộp bài dạng tệp (upload server-side + confirm submitted). \\ \hline

\texttt{seq-quiz-attempt-submit} &
\texttt{POST /api/students/assignments/*/attempts/start}, \texttt{POST /api/students/assignments/*/submit} &
\texttt{AssignmentAttempt, AssignmentSubmission, Question, Option, Notification} &
Attempt + submit quiz + auto-grade. \\ \hline

\texttt{seq-exam-events} &
\texttt{POST|GET /api/exam-events} &
\texttt{ExamEvent, Assignment, Notification} &
Ghi nhận và truy vấn sự kiện anti-cheat. \\ \hline

\texttt{seq-anti-cheat-ai-summary} &
\texttt{POST /api/ai/anti-cheat/summary} &
\texttt{ExamEvent, Assignment} &
AI summary dựa trên events + scoring. \\ \hline

\texttt{seq-announcement-post-and-notify} &
\texttt{GET|POST /api/classrooms/*/announcements} &
\texttt{Announcement, ClassroomStudent, Notification} &
Teacher post announcement + notify students. \\ \hline

\texttt{seq-announcement-comments-and-replies} &
\texttt{GET|POST /api/announcements/*/comments} &
\texttt{Announcement, AnnouncementComment, Notification} &
Comment/reply announcement + notifications. \\ \hline

\texttt{seq-announcement-attachments} &
\texttt{GET|POST /api/announcements/*/attachments}, \texttt{GET /api/announcements/attachments/*/download} &
\texttt{AnnouncementAttachment, Announcement} &
Upload attachment + signed URL download. \\ \hline

\texttt{seq-assignment-attachments} &
\texttt{POST /api/assignments/*/upload}, \texttt{GET /api/assignments/*/files} &
\texttt{AssignmentFile, Assignment} &
Upload attachment cho bài tập + list signed URL. \\ \hline

\texttt{seq-chat-attachments} &
\texttt{POST /api/chat/conversations/*/attachments}, \texttt{GET|POST /api/chat/messages} &
\texttt{Conversation, ConversationParticipant, Message, ChatAttachment, Notification} &
Chat + tệp đính kèm + notify phụ huynh (trong một số trường hợp). \\ \hline

\texttt{seq-notifications} &
\texttt{GET /api/notifications}, \texttt{PATCH /api/notifications/*}, \texttt{POST /api/notifications/mark-all-read} &
\texttt{Notification} &
List/mark read/mark all read. \\ \hline

\texttt{seq-parent-link-and-weekly-summary} &
\texttt{POST /api/parent/link-requests}, \texttt{GET /api/student/link-requests}, \texttt{POST /api/parent/invitations/accept}, \texttt{GET|POST /api/cron/parent-weekly-summary} &
\texttt{ParentStudent, ParentStudentLinkRequest, ParentStudentInvitation, Notification, AuditLog} &
Liên kết + cron summary + notify. \\ \hline

\texttt{seq-reset-password} &
\texttt{POST /api/auth/reset-password/send-code}, \texttt{POST /api/auth/reset-password/verify-code}, \texttt{POST /api/auth/reset-password/reset} &
\texttt{PasswordReset, User, AuditLog} &
Khôi phục mật khẩu. \\ \hline

\texttt{seq-cron-index-lesson-embeddings} &
\texttt{GET|POST /api/cron/index-lesson-embeddings} &
\texttt{Lesson, LessonEmbeddingChunk, AuditLog} &
Cron indexing embeddings bài giảng. \\ \hline

\texttt{state-file-submission} &
\texttt{POST|PUT /api/submissions} &
\texttt{Submission} &
Trạng thái draft/submitted cho nộp bài dạng tệp. \\ \hline

\texttt{state-assignment-attempt} &
\texttt{POST /api/students/assignments/*/attempts/start}, \texttt{POST /api/teachers/assignments/*/attempts/override}, \texttt{POST /api/students/assignments/*/submit} &
\texttt{AssignmentAttempt} &
Trạng thái attempt quiz. \\ \hline

\end{tabularx}
\caption{Mapping Sơ đồ UML $\leftrightarrow$ API $\leftrightarrow$ Prisma Models}
\label{tab:diagram-api-model-mapping}
\end{table}

\chapter{HIỆN THỰC HÓA HỆ THỐNG}

\section{Mô-đun AI Tutor theo kiến trúc RAG}
Phần này trình bày hiện thực mô-đun AI Tutor trong EduVerse theo hướng Retrieval-Augmented Generation (RAG). Trong phiên bản triển khai, hệ thống xây dựng pipeline nội bộ để kiểm soát chunking, indexing và truy vấn ngữ nghĩa trên cùng hạ tầng PostgreSQL.

\subsection{Chuẩn bị dữ liệu và chunking nội dung bài học}
Nguồn tri thức của AI Tutor là dữ liệu bài học (lesson) do giáo viên biên soạn. Trước khi tạo embedding, hệ thống ghép nội dung theo dạng \texttt{\# <title>\textbackslash n\textbackslash n<content>} và đưa qua bước chunking với tham số \texttt{maxChars} (mặc định 1200). Thuật toán ưu tiên tách theo đoạn (phân cách bởi \texttt{\textbackslash n\textbackslash n}); nếu một đoạn vẫn vượt ngưỡng, hệ thống fallback sang tách theo từ để đảm bảo mỗi chunk không vượt quá giới hạn, qua đó cân bằng giữa độ phủ ngữ cảnh và chi phí tính toán embedding.

\begin{figure}[H]
    \centering
    \IfFileExists{figures/uml/flow-rag.png}{
        \includegraphics[width=0.95\textwidth,height=0.8\textheight,keepaspectratio]{figures/uml/flow-rag.png}
    }{
        \IfFileExists{docs/luanvan/figures/uml/flow-rag.png}{
            \includegraphics[width=0.95\textwidth,height=0.8\textheight,keepaspectratio]{docs/luanvan/figures/uml/flow-rag.png}
        }{
            \framebox{\parbox{0.95\textwidth}{\centering\vspace{1cm}Activity/Flow Diagram (PlantUML PNG)\vspace{1cm}}}
        }
    }
    \caption{Quy trình xử lý AI Tutor theo kiến trúc RAG}
    \label{fig:flow-rag}
\end{figure}

\subsection{Tạo embedding bằng Gemini Embedding}
Đối với mỗi chunk, hệ thống gọi Gemini Embedding model \texttt{gemini-embedding-001} \cite{gemini} để tạo vector biểu diễn ngữ nghĩa. Embedding được tiêu chuẩn hóa về số chiều \texttt{1536}. Trong hiện thực, chunk bài học sử dụng task type \texttt{RETRIEVAL\_DOCUMENT} và câu hỏi học sinh sử dụng \texttt{RETRIEVAL\_QUERY} để tối ưu truy vấn.

Hệ thống tăng độ tin cậy (reliability) khi gọi dịch vụ bên ngoài bằng cách phân loại một số lỗi tạm thời (ví dụ HTTP 429/rate limit, timeout, 502/503) vào nhóm \textit{retryable errors} và thực hiện retry theo cơ chế \textbf{exponential backoff} (thời gian chờ tăng theo lũy thừa và có ngưỡng trần), nhằm giảm nguy cơ tạo tải đột biến và tuân thủ hạn mức dịch vụ. Đồng thời, hệ thống kiểm tra ràng buộc số chiều embedding (\texttt{1536}) để tránh ghi dữ liệu sai định dạng vào cột vector.

\subsection{Lưu trữ embedding và truy vấn vector similarity bằng pgvector}
Embedding sau khi tạo được lưu vào bảng \texttt{lesson\_embedding\_chunks}. Mỗi chunk được định danh ổn định theo cặp khóa \texttt{(lessonId, chunkIndex)}. Để đảm bảo \textbf{tính lũy đẳng (idempotency)} của pipeline indexing, hệ thống tính \texttt{contentHash} (SHA-256) cho từng chunk và bỏ qua việc tạo embedding nếu hash không đổi. Khi cần ghi dữ liệu, hệ thống thực hiện upsert bằng câu lệnh \texttt{INSERT ... ON CONFLICT (lessonId, chunkIndex) DO UPDATE}, nhờ đó chạy lại nhiều lần vẫn không tạo trùng bản ghi mà chỉ cập nhật khi nội dung thay đổi.

\begin{lstlisting}[language=JavaScript,caption={Cơ chế contentHash và upsert trong pipeline indexing (trích từ src/lib/rag/indexLessonEmbeddings.ts)},label={lst:code-contenthash-upsert}]
function sha256Hex(text: string): string {
  return crypto.createHash("sha256").update(text, "utf8").digest("hex");
}

const contentHash = sha256Hex(ch.content);
const vec = toVectorLiteral(embedding);
const id = `lec_${lessonId}_${ch.index}`;

await prisma.$executeRaw`
  INSERT INTO "lesson_embedding_chunks" (
    "id", "lessonId", "courseId", "chunkIndex", "content", "contentHash", "embedding", "updatedAt"
  )
  VALUES (
    ${id}, ${lessonId}, ${courseId}, ${ch.index}, ${ch.content}, ${contentHash}, ${vec}::vector, NOW()
  )
  ON CONFLICT ("lessonId", "chunkIndex")
  DO UPDATE SET
    "content" = EXCLUDED."content",
    "contentHash" = EXCLUDED."contentHash",
    "embedding" = EXCLUDED."embedding",
    "updatedAt" = NOW();
`;
\end{lstlisting}

Ngoài ra, hệ thống có cơ chế \textbf{garbage collection} cho dữ liệu embedding: khi nội dung bài giảng bị rút ngắn làm giảm số lượng chunk, các chunk ``dư thừa'' có \texttt{chunkIndex} lớn hơn chunk cuối cùng sẽ được xóa khỏi bảng embedding. Cách làm này giúp đảm bảo dữ liệu vector trong DB luôn phản ánh đúng phiên bản nội dung mới nhất và tránh gây nhiễu khi truy vấn Top-$k$.

Khi học sinh đặt câu hỏi, hệ thống:
\begin{enumerate}
    \item tạo embedding cho câu hỏi;
    \item truy vấn các chunk gần nhất theo khoảng cách vector trong PostgreSQL;
    \item ghép các chunk Top-$k$ vào prompt làm ngữ cảnh trả lời.
\end{enumerate}

Truy vấn similarity sử dụng toán tử khoảng cách của pgvector dưới dạng \texttt{("embedding" <=> queryVector::vector) AS distance}, sau đó sắp xếp \texttt{ORDER BY distance ASC} và lấy \texttt{LIMIT topK}. Kết quả là danh sách các đoạn nguồn được dùng để ràng buộc câu trả lời của AI Tutor, nhằm giảm hiện tượng “ảo giác” \cite{rag}.

\subsection{Tổ chức pipeline indexing (teacher-trigger và cron)}
Hệ thống hỗ trợ hai cách kích hoạt indexing:
\begin{itemize}

    \item \textbf{Teacher-trigger:} giáo viên gọi endpoint indexing theo khóa học để tạo hoặc làm mới embedding khi cập nhật lesson.
    \item \textbf{Cron indexing:} tiến trình định kỳ quét lesson cập nhật và thực hiện indexing, có xác thực bằng bí mật cron.
\end{itemize}

\section{Hệ thống đánh giá và giám sát thi cử}
\subsection{Cơ chế tự động chấm bài}
Đối với bài tự luận (ESSAY), hệ thống hỗ trợ giáo viên bằng cơ chế gợi ý chấm điểm: server lấy nội dung bài nộp, kết hợp thông tin đề bài và rubric, sau đó gọi Gemini để sinh gợi ý điểm số và nhận xét. Các API AI được áp dụng giới hạn tần suất (rate limit) nhằm giảm nguy cơ lạm dụng tài nguyên dịch vụ AI.

\begin{figure}[H]
    \centering
    \IfFileExists{figures/uml/seq-file-submission.png}{
        \includegraphics[width=0.95\textwidth,height=0.8\textheight,keepaspectratio]{figures/uml/seq-file-submission.png}
    }{
        \IfFileExists{docs/luanvan/figures/uml/seq-file-submission.png}{
            \includegraphics[width=0.95\textwidth,height=0.8\textheight,keepaspectratio]{docs/luanvan/figures/uml/seq-file-submission.png}
        }{
            \framebox{\parbox{0.95\textwidth}{\centering\vspace{1cm}Sequence Diagram (PlantUML PNG)\vspace{1cm}}}
        }
    }
    \caption{Quy trình nộp bài dạng tệp}
    \label{fig:seq-file-submission}
\end{figure}

\subsection{Phát hiện hành vi đáng ngờ và tổng hợp báo cáo chống gian lận (Anti-cheat)}
Trong EduVerse, dữ liệu chống gian lận được thu thập dưới dạng \textbf{event log} trong quá trình học sinh làm bài quiz. Các sự kiện được ghi nhận tại giao diện thi (client) và lưu về server để phục vụ phân tích.

\subsubsection{Thu thập dữ liệu: ánh xạ hành vi trình duyệt sang exam events}
Các tín hiệu gian lận trọng yếu trong hiện thực bao gồm:
\begin{itemize}

    \item \textbf{Chuyển tab (visibilitychange):} khi \texttt{document.hidden = true}, hệ thống ghi event \texttt{TAB\_SWITCH\_DETECTED} với nguồn \texttt{visibilitychange}.
    \item \textbf{Mất focus cửa sổ (blur):} khi \texttt{window.blur} xảy ra, hệ thống có thể ghi event \texttt{TAB\_SWITCH\_DETECTED} với nguồn \texttt{window\_blur}.
    \item \textbf{Rời cửa sổ (WINDOW\_BLUR):} trong một số luồng giao diện thi khác, client có thể ghi trực tiếp event \texttt{WINDOW\_BLUR} khi phát hiện \texttt{window.blur}. Tín hiệu này phản ánh hành vi chuyển cửa sổ/ứng dụng trong lúc làm bài.
    \item \textbf{Clipboard/context menu:} hệ thống chặn thao tác copy/paste và ghi event \texttt{COPY\_PASTE\_ATTEMPT} với nguồn \texttt{contextmenu} hoặc phím tắt tương ứng.
    \item \textbf{Thoát fullscreen:} khi bắt buộc fullscreen và học sinh thoát fullscreen, hệ thống ghi event \texttt{FULLSCREEN\_EXIT}.
\end{itemize}

Để đảm bảo \textbf{tính nhất quán} khi có nhiều biến thể ghi log từ client, server thực hiện bước \textbf{chuẩn hóa (normalization)} \texttt{eventType} trước khi tính điểm. Cụ thể, các event \texttt{TAB\_SWITCH\_DETECTED} (dù được phát sinh từ \texttt{visibilitychange} hay \texttt{window\_blur}) đều được gom về nhóm \texttt{TAB\_SWITCH}. Tương tự, \texttt{COPY\_PASTE\_ATTEMPT} được chuẩn hóa về nhóm \texttt{CLIPBOARD}. Việc chuẩn hóa này giúp công thức tính \texttt{suspicionScore} không phụ thuộc vào chi tiết hiện thực UI/trình duyệt, đồng thời tạo điều kiện so sánh điểm số ổn định theo thời gian.

\begin{lstlisting}[language=JavaScript,caption={Chuẩn hóa sự kiện và tính điểm suspicionScore (trích từ src/lib/exam-session/antiCheatScoring.ts)},label={lst:code-anti-cheat-score}]
function normalizeEventType(eventType: string): string {
  const raw = (eventType || "").toString().trim();
  if (!raw) return "";

  switch (raw) {
    case "TAB_SWITCH_DETECTED":
      return "TAB_SWITCH";
    case "COPY_PASTE_ATTEMPT":
      return "CLIPBOARD";
    default:
      return raw;
  }
}

export function computeQuizAntiCheatScore(events: ExamEventForScoring[]): AntiCheatScoreResult {
  const countsByType: Record<string, number> = {};

  for (const ev of events) {
    const type = normalizeEventType(ev.eventType);
    if (!type) continue;
    countsByType[type] = (countsByType[type] ?? 0) + 1;
  }

  // ... rule-based breakdown + clamp score 0..100
}
\end{lstlisting}

\subsubsection{Thuật toán tính điểm: suspicionScore theo rule-based scoring}
Server tính điểm nghi ngờ \texttt{suspicionScore} theo cơ chế rule-based, ánh xạ từng loại event sang một rule với trọng số và mức trần điểm. Mỗi rule có dạng:
\begin{quote}
\texttt{points = min(maxPoints, count * pointsPerHit)}
\end{quote}
Điểm tổng được chặn trong khoảng $[0,100]$. Trong hiện thực, các rule trọng yếu và tham số tương ứng được tổng hợp trong Bảng \ref{tab:anti-cheat-weights}.

\begin{table}[H]
\centering
\small
\begin{tabularx}{\textwidth}{|X|>{\centering\arraybackslash}p{3cm}|>{\centering\arraybackslash}p{3cm}|}
\hline
\rowcolor[gray]{0.9} \textbf{Quy tắc / Nhóm sự kiện (sau chuẩn hóa)} & \textbf{pointsPerHit} & \textbf{maxPoints} \\ \hline
Thoát toàn màn hình (\texttt{FULLSCREEN\_EXIT}) & 20 & 40 \\ \hline
Chuyển tab trình duyệt (\texttt{TAB\_SWITCH}) & 12 & 60 \\ \hline
Rời khỏi cửa sổ (\texttt{WINDOW\_BLUR}) & 5 & 20 \\ \hline
Thao tác Clipboard (\texttt{CLIPBOARD}) & 8 & 24 \\ \hline
Phím tắt đáng ngờ (\texttt{SHORTCUT}) & 6 & 18 \\ \hline
\end{tabularx}
\caption{Bảng trọng số tính điểm \texttt{suspicionScore} theo cơ chế rule-based}
\label{tab:anti-cheat-weights}
\end{table}
 
 \begin{figure}[H]
     \centering
     \IfFileExists{docs/luanvan/figures/uml/flow-anti-cheat.png}{
         \includegraphics[width=0.95\textwidth,height=0.8\textheight,keepaspectratio]{docs/luanvan/figures/uml/flow-anti-cheat.png}
     }{
         \framebox{\parbox{0.95\textwidth}{\centering\vspace{1cm}Activity/Flow Diagram (PlantUML PNG)\vspace{1cm}}}
     }
     \caption{Luồng xử lý chống gian lận: scoring và AI summary}
     \label{fig:flow-anti-cheat}
 \end{figure}
 
\chapter{ĐÁNH GIÁ VÀ THỬ NGHIỆM}
 
 \section{Môi trường thử nghiệm và phương pháp đánh giá}
 Hệ thống EduVerse được kiểm thử theo hướng kết hợp giữa:
 \begin{itemize}
     \item \textbf{Kiểm thử đơn vị (Unit Test):} sử dụng Vitest (script \texttt{vitest run}) để kiểm tra các hàm nghiệp vụ quan trọng như chia đoạn văn bản cho RAG (\texttt{chunkText}) và tính điểm nghi ngờ anti-cheat (\texttt{suspicionScore}).
     \item \textbf{Đánh giá luồng API:} đánh giá luồng xử lý của các endpoint chính (RAG Tutor chat, ghi exam events, anti-cheat scoring và AI summary) dựa trên ràng buộc dữ liệu và điều kiện xác thực/giới hạn tần suất.
     \item \textbf{Đánh giá trải nghiệm người dùng (UX):} tập trung vào tính nhất quán của luồng thao tác theo vai trò và cơ chế phản hồi trạng thái ở các thao tác quan trọng.
 \end{itemize}
 
 \section{Kiểm thử chức năng}
 Bảng \ref{tab:testcases-core} tổng hợp 5 kịch bản kiểm thử chức năng quan trọng, tập trung vào các luồng có rủi ro cao trong vận hành thực tế (xác thực theo vai trò, nộp bài, AI Tutor và chống gian lận).
 
\begin{table}[H]
\centering
\small
\begingroup
\setlength{\tabcolsep}{3pt}
\renewcommand{\arraystretch}{1.15}
\begin{tabularx}{\textwidth}{|p{0.9cm}|p{2.8cm}|X|X|X|}
\hline
\textbf{TC} & \textbf{Chức năng} & \textbf{Tiền điều kiện} & \textbf{Các bước thực hiện (tóm tắt)} & \textbf{Kỳ vọng} \\
\hline
TC01 & Xác thực portal theo vai trò (RBAC) & Có tài khoản Teacher/Student; phiên đăng nhập hợp lệ & (1) Đăng nhập. (2) Gọi API/đi tới trang thuộc vai trò khác. & Bị từ chối với HTTP 403; chỉ role hợp lệ được phép truy cập. \\
\hline
TC02 & Nộp bài tập (AssignmentSubmission) & Student thuộc Classroom có Assignment; attempt hợp lệ & (1) Student gửi bài nộp. (2) Kiểm tra DB có bản ghi \texttt{(assignmentId, studentId, attempt)}. & Tạo bản nộp thành công; ràng buộc unique tránh trùng attempt. \\
\hline
TC03 & AI Tutor phản hồi theo dữ liệu RAG & Student là thành viên lớp; lesson đã/hoặc chưa index embedding & (1) Gửi POST . (2) Quan sát câu trả lời và danh sách nguồn. & Nếu có embedding: trả về \texttt{answer + sources}. Nếu chưa có embedding: trả về thông báo và cờ \texttt{noEmbeddings}. \\
\hline
TC04 & Anti-cheat ghi nhận log sự kiện & Student đang làm Quiz; có \texttt{assignmentId} & (1) Client phát hiện sự kiện (tab switch/fullscreen exit). (2) Gửi POST /api/exam-events. & DB tạo bản ghi \texttt{exam\_events}; metadata bị giới hạn kích thước. \\
\hline
TC05 & Giáo viên xem điểm nghi ngờ và AI summary & Teacher là chủ Assignment (Quiz); có dữ liệu exam events & (1) Teacher xem điểm nghi ngờ. (2) Gọi /api/ai/anti-cheat/summary. & Trả về \texttt{suspicionScore}, \texttt{riskLevel}; AI summary chỉ áp dụng cho Quiz và có rate-limit. \\
\hline
\end{tabularx}
\endgroup
\caption{Các kịch bản kiểm thử chức năng cốt lõi của EduVerse}
\label{tab:testcases-core}
\end{table}

\section{Đánh giá độ chính xác và hiệu năng của AI Tutor}
Luồng RAG Tutor (\texttt{/api/ai/tutor/chat}) gồm ba pha chính: (i) kiểm tra xác thực và thành viên lớp, (ii) tạo embedding truy vấn và truy vấn pgvector để lấy Top-$k$ chunk gần nhất, (iii) gọi mô hình sinh nội dung để tạo câu trả lời có ràng buộc nguồn tham khảo.

Về kiểm soát tải và độ ổn định, endpoint áp dụng \textbf{rate-limit hai lớp}:
\begin{itemize}
    \item Theo IP: giới hạn 20 yêu cầu / 10 phút.
    \item Theo người dùng (student): giới hạn 20 yêu cầu / 10 phút.
\end{itemize}
Khi vượt ngưỡng, hệ thống trả về HTTP 429 kèm \texttt{Retry-After}. Ngoài ra, hệ thống giới hạn kích thước message và lịch sử hội thoại để giảm chi phí, đồng thời giới hạn độ dài phản hồi (\texttt{maxOutputTokens}) nhằm ổn định độ trễ khi số lượng người dùng tăng.

\section{Đánh giá anti-cheat: scoring và AI summary}
Đối với chống gian lận, hệ thống tách hai tầng:
\begin{itemize}
    \item \textbf{Tầng rule-based scoring:} tính \texttt{suspicionScore} trong khoảng $[0,100]$ dựa trên tần suất sự kiện và cơ chế \textit{cap điểm theo từng quy tắc}. Ví dụ: \texttt{FULLSCREEN\_EXIT} có 20 điểm/lần nhưng tối đa 40 điểm; \texttt{TAB\_SWITCH} có 12 điểm/lần tối đa 60 điểm.
    \item \textbf{Tầng AI summary:} endpoint \texttt{/api/ai/anti-cheat/summary} chỉ áp dụng cho Quiz và lấy tối đa 250 sự kiện theo thời gian, sau đó tạo tóm tắt dựa trên \texttt{suspicionScore}, \texttt{riskLevel} và breakdown. Endpoint áp dụng rate-limit theo IP và theo teacher nhằm đảm bảo chi phí và độ ổn định.
\end{itemize}
Thiết kế hai tầng giúp hệ thống giữ được tính quyết định của điểm nghi ngờ (phục vụ so sánh/đối chiếu) đồng thời dùng AI để tăng khả năng diễn giải bằng ngôn ngữ tự nhiên.

\section{Đánh giá trải nghiệm người dùng (UI/UX)}
EduVerse tổ chức giao diện theo portal đa vai trò (Teacher/Student/Parent/Admin) và áp dụng cơ chế phản hồi UI nhất quán.

Ở Teacher dashboard, dữ liệu được tải qua cơ chế caching/revalidate và có \textbf{skeleton loading} khi chưa có dữ liệu, \textbf{error state} rõ ràng khi lỗi và \textbf{empty state} có hướng dẫn hành động.

Ở giao diện chat của học sinh, hệ thống có cơ chế xử lý rate-limit thân thiện (hiển thị dialog kèm thời gian chờ) và có bước chuẩn hóa nội dung câu trả lời để tránh lộ thông tin kỹ thuật. Ngoài ra, hội thoại được lưu theo bài học để đảm bảo tính liên tục khi người dùng quay lại.

\section{Kết quả thực nghiệm}
Phần này trình bày kết quả đo thực nghiệm ở mức hệ thống (end-to-end) theo ba kịch bản chính: S1 (AI Tutor chat), S2 (ghi \texttt{ExamEvent}) và S3 (AI Anti-cheat Summary). Dữ liệu được thu thập bằng cơ chế theo dõi hiệu năng ở phía server và xuất ra theo khoảng thời gian 60 phút.

Bảng \ref{tab:experiment-results} tổng hợp các chỉ số đại diện theo từng endpoint: số request, độ trễ trung bình, P95, độ trễ lớn nhất và tỷ lệ lỗi.

\begin{table}[H]
\centering
\small
\begingroup
\setlength{\tabcolsep}{3pt}
\renewcommand{\arraystretch}{1.15}
\begin{tabularx}{\textwidth}{|p{4.2cm}|X|p{3.2cm}|}
\hline
\textbf{Hạng mục} & \textbf{Mô tả đo lường} & \textbf{Kết quả (đo thực nghiệm)} \\ \hline
API RAG Tutor  & Thời gian phản hồi end-to-end từ lúc gửi câu hỏi đến lúc nhận câu trả lời; bao gồm tạo embedding truy vấn, truy vấn pgvector Top-$k$ và sinh nội dung. & n=10; Avg 5.2s; P95 5.5s; Max 7.4s; Error 0\% \\ \hline
Thu thập \texttt{ExamEvent}  & Độ trễ ghi nhận sự kiện làm bài (server-side) và tỷ lệ request ghi log thành công. & n=200; Avg 0.8s; P95 1.4s; Max 1.5s; Error 0\% \\ \hline
AI Anti-cheat Summary  & Thời gian sinh tóm tắt chống gian lận dựa trên \texttt{suspicionScore}/\texttt{riskLevel} và tối đa 250 sự kiện theo attempt. & n=8; Avg 5.6s; P95 6.3s; Max 6.9s; Error 0\% \\ \hline
\end{tabularx}
\endgroup
\caption{Tóm tắt kết quả đo thực nghiệm (end-to-end) của hệ thống EduVerse}
\label{tab:experiment-results}
\end{table}
\noindent Trong kịch bản S2 (thu thập \texttt{ExamEvent}), hệ thống ghi nhận 200/200 yêu cầu ghi log thành công, tương ứng successRate xấp xỉ 100\% trong lần đo.

\section{Hạn chế}
Mặc dù hệ thống đã đáp ứng các luồng nghiệp vụ cốt lõi, EduVerse vẫn tồn tại một số hạn chế cần được xem xét khi triển khai thực tế.

\textbf{(1) Độ trễ phụ thuộc dịch vụ AI bên ngoài.} Các tác vụ tạo embedding, sinh câu trả lời RAG, tạo tóm tắt anti-cheat và gợi ý chấm bài đều phụ thuộc Gemini API, do đó độ trễ end-to-end có thể dao động theo chất lượng mạng, tải hệ thống phía nhà cung cấp và hạn mức (rate limit). Điều này ảnh hưởng trực tiếp đến trải nghiệm tương tác thầm gian thực của học sinh và tốc độ xử lý nghiệp vụ của giáo viên.

\textbf{(2) Chi phí token và chi phí vận hành.} Kiến trúc RAG phát sinh chi phí ở cả hai pha: indexing (tạo embedding cho học liệu) và inference (sinh nội dung theo ngữ cảnh). Mặc dù pipeline đã tối ưu bằng cơ chế \texttt{contentHash} để tránh index lại nội dung không đổi, chi phí tổng vẫn có thể tăng đáng kể khi số lượng lớp học, bài giảng và truy vấn tăng.

\textbf{(3) Phụ thuộc chất lượng học liệu đầu vào.} RAG giúp bám sát tài liệu nhưng chất lượng phản hồi phụ thuộc trực tiếp vào cấu trúc và độ rõ ràng của học liệu. Nếu nội dung bài giảng thiếu mạch lạc hoặc có thông tin sai lệch, các đoạn truy xuất có thể không phù hợp, làm giảm chất lượng câu trả lời.

\textbf{(4) Giới hạn của anti-cheat dựa trên event log.} Cơ chế scoring hiện tại chủ yếu dựa trên tín hiệu trình duyệt và các quy tắc (rule-based), do đó vẫn có nguy cơ false positive/false negative. Bản tóm tắt AI hỗ trợ diễn giải nhưng không thay thế hoàn toàn vai trò ra quyết định của giáo viên và quy chế của nhà trường.

% ======================================================
% CHƯƠNG 6: KẾT LUẬN VÀ HƯỚNG PHÁT TRIỂN
% ======================================================

\chapter{KẾT LUẬN VÀ HƯỚNG PHÁT TRIỂN}

\section{Kết quả đạt được}
Luận văn đã nghiên cứu và hiện thực hóa hệ thống \textbf{EduVerse} theo định hướng Smart LMS đa vai trò, trong đó trọng tâm là tích hợp AI theo kiến trúc RAG và tăng cường giám sát liêm chính học thuật.

Các kết quả đạt được có thể tóm tắt như sau:
\begin{itemize}
    \item \textbf{Kiến trúc triển khai full-stack thống nhất:} ứng dụng được hiện thực hóa bằng Next.js App Router theo mô hình tổ chức UI và API trong cùng codebase, giúp các tác vụ nhạy cảm (truy vấn CSDL, gọi Gemini API, xử lý phân quyền) được thực thi phía server.
    \item \textbf{Xác thực và phân quyền theo RBAC:} hệ thống sử dụng NextAuth (JWT strategy) và middleware để đảm bảo người dùng truy cập đúng portal theo vai trò, đồng thời các API Routes kiểm tra quyền sở hữu dữ liệu và vai trò trước khi xử lý.
    \item \textbf{AI Tutor theo kiến trúc RAG:} xây dựng pipeline chunking, indexing và truy vấn vector trên PostgreSQL + pgvector; sử dụng \texttt{gemini-embedding-001} cho embedding và \texttt{gemini-2.5-flash-lite} cho sinh câu trả lời có ràng buộc ngữ cảnh.
    \item \textbf{Tối ưu indexing theo hướng lũy đẳng:} cơ chế \texttt{contentHash} kết hợp upsert giúp tránh tạo embedding lại khi nội dung không đổi, góp phần giảm chi phí và tăng tính ổn định khi vận hành.
    \item \textbf{Giám sát thi và chống gian lận:} thu thập event log, chuẩn hóa \texttt{eventType}, tính \texttt{suspicionScore} theo rule-based scoring và cung cấp tóm tắt diễn giải bằng AI để hỗ trợ giáo viên rà soát nhanh.
    \item \textbf{Hỗ trợ chấm bài tự luận:} hệ thống cung cấp gợi ý chấm điểm/nhận xét dựa trên bài nộp và rubric, giúp giảm tải các tác vụ lặp lại cho giáo viên.
\end{itemize}

Các kết quả trên cho thấy hướng tiếp cận tích hợp RAG và LLM vào hệ thống LMS theo hướng ưu tiên tính kiểm chứng, khả năng truy vết và kiểm soát chi phí. Các kết quả đạt được là nền tảng cho việc tiếp tục nghiên cứu, hoàn thiện và triển khai trong bối cảnh giáo dục thực tế.

\section{Đối chiếu với mục tiêu nghiên cứu}
So với các mục tiêu đề ra ở Chương 1, hệ thống đáp ứng được các yêu cầu chính:
\begin{itemize}
    \item \textbf{Kiến trúc LMS đa vai trò:} mô hình multi-portal theo RBAC giúp phân tách luồng thao tác theo nhóm người dùng và giảm nguy cơ truy cập nhầm chức năng.
    \item \textbf{AI Tutor dựa trên CSDL vector:} pipeline RAG được hiện thực hóa đầy đủ từ indexing học liệu đến truy vấn Top-$k$ và sinh câu trả lời, đồng thời có cơ chế kiểm soát truy cập theo lesson/course.
    \item \textbf{Tự động hóa đánh giá:} cơ chế gợi ý chấm bài tự luận hỗ trợ giáo viên tăng tốc phản hồi, có thể mở rộng theo hướng bán tự động với kiểm soát của con người (human-in-the-loop).
    \item \textbf{Liêm chính học thuật:} scoring dựa trên event log kết hợp AI summary hỗ trợ phát hiện/diễn giải hành vi đáng ngờ, nhưng vẫn cần kết hợp quy chế và quyết định của giáo viên.
\end{itemize}

\section{Hướng phát triển tương lai}
Trong tương lai, hệ thống có thể được nâng cấp theo các hướng sau:
\begin{itemize}
    \item \textbf{Multimodal AI Tutor:} mở rộng khả năng hiểu và trả lời dựa trên hình ảnh/biểu đồ, hoặc tài liệu PDF có cấu trúc, giúp hỗ trợ tốt hơn các môn học có nội dung trực quan.
    \item \textbf{Proctoring nâng cao:} kết hợp thêm tín hiệu ngoài trình duyệt (ví dụ video/âm thanh) khi có cơ chế đồng thuận và chính sách bảo mật rõ ràng, nhằm giảm các trường hợp gian lận ngoài phạm vi quan sát của event log.
    \item \textbf{Tối ưu chi phí và độ trễ:} áp dụng cache theo câu hỏi phổ biến, tóm tắt ngữ cảnh theo phiên, điều chỉnh Top-$k$ động và giám sát chi phí token theo lớp/môn để tối ưu vận hành.
    \item \textbf{Đánh giá chất lượng AI có hệ thống:} xây dựng bộ benchmark theo từng môn và các tiêu chí faithfulness/groundedness để theo dõi chất lượng câu trả lời RAG theo thời gian.
    \item \textbf{Mở rộng trải nghiệm đa nền tảng:} phát triển ứng dụng di động, thông báo đẩy và cá nhân hóa lộ trình học dựa trên tiến độ và lịch sử tương tác.
\end{itemize}

\section{Kết luận}
EduVerse chứng minh tính khả thi của việc tích hợp RAG và LLM vào hệ thống LMS theo hướng ưu tiên tính kiểm chứng, khả năng truy vết và kiểm soát chi phí. Các kết quả đạt được là nền tảng cho việc tiếp tục nghiên cứu, hoàn thiện và triển khai trong bối cảnh giáo dục thực tế.

% ======================================================
% TÀI LIỆU THAM KHẢO
% ======================================================
\begin{thebibliography}{99}
\addcontentsline{toc}{chapter}{Tài liệu tham khảo}
\bibitem{nextjs} Vercel, "Next.js 14 Documentation," 2024. [Online]. Available: \url{https://nextjs.org/docs}.
\bibitem{nextauth} NextAuth.js, "NextAuth.js Documentation," 2024. [Online]. Available: \url{https://next-auth.js.org}.
\bibitem{prisma} Prisma, "Prisma Documentation," 2024. [Online]. Available: \url{https://www.prisma.io/docs}.
\bibitem{postgresql} The PostgreSQL Global Development Group, "PostgreSQL Documentation," 2024. [Online]. Available: \url{https://www.postgresql.org/docs/}.
\bibitem{pgvector} pgvector, "pgvector: Open-source vector similarity search for Postgres," 2024. [Online]. Available: \url{https://github.com/pgvector/pgvector}.
\bibitem{supabase} Supabase, "Supabase Documentation," 2024. [Online]. Available: \url{https://supabase.com/docs}.
\bibitem{plantuml} PlantUML, "PlantUML Documentation," 2024. [Online]. Available: \url{https://plantuml.com/}.
\bibitem{gemini} Google DeepMind, "Gemini: A Family of Highly Capable Multimodal Models," 2023.
\bibitem{rag} Lewis, P., et al., "Retrieval-Augmented Generation for Knowledge-Intensive NLP Tasks," NeurIPS, 2020.
\end{thebibliography}

\end{document}