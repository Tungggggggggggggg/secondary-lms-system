\documentclass[12pt, a4paper]{report}

% --- UNIVERSAL PREAMBLE BLOCK FOR VIETNAMESE ---
\usepackage[a4paper, top=20mm, bottom=20mm, left=35mm, right=20mm]{geometry}
\usepackage{fontspec}
\usepackage[bidi=default, provide=*]{babel}

% Cấu hình ngôn ngữ và font chữ (Sử dụng Noto Serif để thay thế Times New Roman trong môi trường này)
\babelprovide[main, import]{vietnamese}
\babelprovide[import]{english}


\IfFontExistsTF{Times New Roman}{
\babelfont{rm}{Times New Roman}
\babelfont[vietnamese]{rm}{Times New Roman}
}{
\babelfont{rm}{Latin Modern Roman}
\babelfont[vietnamese]{rm}{Latin Modern Roman}
}


% Fix cho danh sách khi sử dụng ngôn ngữ không phải tiếng Anh
\usepackage{enumitem}
\setlist[itemize]{label=-}
% --- GÓI BỔ TRỢ HỌC THUẬT ---
\usepackage{amsmath, amsfonts, amssymb}
\usepackage{graphicx}
\graphicspath{{docs/luanvan/}{docs/luanvan/figures/}{docs/luanvan/figures/uml/}{figures/}{figures/uml/}{uml/}{out/docs/luanvan/figures/}{out/docs/luanvan/figures/uml/}{out/docs/luanvan/figures/uml/usecase-overall/}{out/docs/luanvan/figures/uml/usecase-anti-cheat/}{out/docs/luanvan/figures/uml/sequence-exam-events/}{out/docs/luanvan/figures/uml/activity-suspicious-behavior/}{out/docs/luanvan/figures/uml/context-eduverse/}{out/docs/luanvan/figures/uml/component-architecture/}{out/docs/luanvan/figures/uml/db-core/}}
\usepackage{setspace}
\usepackage{indentfirst}
\usepackage{titlesec}
\usepackage{listings}
\usepackage[table]{xcolor}
\usepackage{booktabs}
\usepackage{array}
\usepackage{tabularx}
\usepackage{caption}
\usepackage{float}
\usepackage{needspace}

% Cấu hình giãn dòng chuẩn luận văn (1.5 line spacing)
\onehalfspacing
\setlength{\parindent}{1.27cm}
\setlength{\emergencystretch}{2em}

% Cấu hình mã nguồn (Code) chuyên nghiệp
\definecolor{codegreen}{rgb}{0,0.6,0}
\definecolor{codegray}{rgb}{0.5,0.5,0.5}
\definecolor{codepurple}{rgb}{0.58,0,0.82}
\definecolor{backcolour}{rgb}{0.95,0.95,0.92}

\lstdefinelanguage{JavaScript}{
    keywords={await,break,case,catch,class,const,continue,debugger,default,delete,do,else,export,extends,false,finally,for,function,if,import,in,instanceof,let,new,null,return,super,switch,this,throw,true,try,typeof,var,void,while,with,yield},
    keywordstyle=\color{magenta},
    sensitive=true,
    morecomment=[l]{//},
    morecomment=[s]{/*}{*/},
    morestring=[b]',
    morestring=[b]",
}

\lstdefinelanguage{TypeScript}[]{JavaScript}{
    keywords={any,as,boolean,constructor,declare,enum,get,module,namespace,number,private,protected,public,readonly,require,set,string,type,undefined,unknown},
}

\lstset{
    backgroundcolor=\color{backcolour},   
    commentstyle=\color{codegreen},
    keywordstyle=\color{magenta},
    numberstyle=\tiny\color{codegray},
    stringstyle=\color{codepurple},
    basicstyle=\ttfamily\footnotesize,
    breakatwhitespace=false,         
    breaklines=true,                 
    captionpos=b,                    
    keepspaces=true,                 
    numbers=left,                    
    numbersep=5pt,                  
    showspaces=false,                
    showstringspaces=false,
    showtabs=false,                  
    tabsize=2
}

% Định dạng tiêu đề chương theo chuẩn: CHƯƠNG 1 (Căn giữa, viết hoa)
\titleformat{\chapter}[display]
  {\normalfont\large\bfseries\centering}
  {\MakeUppercase{\chaptername}\ \thechapter}
  {5pt}
  {\large\MakeUppercase}
\titlespacing*{\chapter}{0pt}{-20pt}{30pt}

% Định dạng tiêu đề mục (Section, Subsection)
\titleformat{\section}{\normalfont\normalsize\bfseries}{\thesection}{1em}{}
\titleformat{\subsection}{\normalfont\normalsize\bfseries\itshape}{\thesubsection}{1em}{}

\usepackage{xurl}
\usepackage{hyperref}
\hypersetup{
    colorlinks=true,
    linkcolor=black,
    filecolor=magenta,      
    urlcolor=blue,
    pdftitle={Luận văn Tốt nghiệp - Smart LMS},
}

\begin{document}

% ======================================================
% TRANG BÌA
% ======================================================
\begin{titlepage}
    \centering
    {\large \textbf{BỘ GIÁO DỤC VÀ ĐÀO TẠO}} \\
    {\large \textbf{TRƯỜNG ĐẠI HỌC GIAO THÔNG VẬN TẢI TP.HCM}} \\
    {\large \textbf{KHOA CÔNG NGHỆ THÔNG TIN}}
    
    \vspace{1.5cm}
    \begin{figure}[h]
      \centering
      \IfFileExists{docs/luanvan/figures/uth-logo.png}{
        \includegraphics[width=0.35\textwidth]{docs/luanvan/figures/uth-logo.png}
      }{
        \IfFileExists{figures/uth-logo.png}{
          \includegraphics[width=0.35\textwidth]{figures/uth-logo.png}
        }{
          \framebox{\parbox{0.35\textwidth}{\centering
            \vspace{1cm}
            \textbf{LOGO TRƯỜNG} \\
            \small\textit{(Image Placeholder)}
            \vspace{1cm}
          }}
        }
      }
    \end{figure}
    
    \vspace{1.5cm}
    {\Large \textbf{LUẬN VĂN TỐT NGHIỆP ĐẠI HỌC}} \\
    
    \vspace{1cm}
    \begin{spacing}{1.5}
        {\huge \textbf{NGHIÊN CỨU VÀ XÂY DỰNG HỆ THỐNG QUẢN LÝ HỌC TẬP THÔNG MINH (SMART LMS) DỰA TRÊN KIẾN TRÚC RAG VÀ MÔ HÌNH NGÔN NGỮ LỚN}}
    \end{spacing}
    
    \vspace{2cm}
    \begin{flushright}
        \begin{tabular}{l p{6cm}}
            \textbf{Giảng viên hướng dẫn:} & \textbf{ThS. Bùi Trọng Hiếu} \\
            \textbf{Sinh viên thực hiện:}     & \textbf{Nguyễn Quốc Tùng} \\
            \textbf{Mã số sinh viên:}         & \textbf{2251120259} \\
            \textbf{Sinh viên thực hiện:}     & \textbf{Nguyễn Võ Thành} \\
            \textbf{Mã số sinh viên:}         & \textbf{2251120246} \\
            \textbf{Lớp:}                     & \textbf{CN22E}
        \end{tabular}
    \end{flushright}

    \vfill
    {\large \textbf{TP. HỒ CHÍ MINH, NĂM 2025}}
\end{titlepage}

% ======================================================
% PHẦN TRƯỚC NỘI DUNG CHÍNH
% ======================================================
\pagenumbering{roman}

\chapter*{Lời cam đoan}
\phantomsection
\addcontentsline{toc}{chapter}{Lời cam đoan}
Chúng tôi xin cam đoan đây là công trình nghiên cứu của nhóm dưới sự hướng dẫn của giảng viên hướng dẫn. Các kết quả nêu trong luận văn là trung thực và chưa từng được công bố trong bất kỳ công trình nào khác. Mọi sự giúp đỡ cho việc thực hiện luận văn này đã được cảm ơn và các thông tin trích dẫn trong luận văn đã được chỉ rõ nguồn gốc.

\chapter*{Lời mở đầu}
\addcontentsline{toc}{chapter}{Lời mở đầu}
Sự bùng nổ của Trí tuệ nhân tạo tạo sinh (Generative AI) đã mở ra những cơ hội chưa từng có trong việc cá nhân hóa giáo dục. Tuy nhiên, các hệ thống Quản lý học tập (LMS) hiện nay tại Việt Nam chủ yếu vẫn đóng vai trò là kho lưu trữ tài liệu tĩnh, thiếu đi tính tương tác thông minh và khả năng hỗ trợ học sinh học tập chủ động.

Dự án này được thực hiện với mong muốn xây dựng một hệ sinh thái học tập "Smart LMS" dành cho cấp trung học. Điểm đột phá của hệ thống là việc ứng dụng kỹ thuật Retrieval-Augmented Generation (RAG) để tạo ra một Trợ lý ảo (AI Tutor) có khả năng hiểu sâu sắc nội dung bài giảng của giáo viên, từ đó phản hồi chính xác thắc mắc của học sinh. Bên cạnh đó, hệ thống tích hợp các công cụ giám sát thi cử dựa trên AI nhằm đảm bảo tính liêm chính trong môi trường học tập trực tuyến.

\newpage

\chapter*{Lời cảm ơn}
\phantomsection
\addcontentsline{toc}{chapter}{Lời cảm ơn}
Trong quá trình thực hiện luận văn và xây dựng hệ thống \textbf{EduVerse}, chúng tôi đã nhận được sự hướng dẫn, hỗ trợ và động viên từ nhiều cá nhân và đơn vị. Chúng tôi xin bày tỏ lòng biết ơn sâu sắc đến \textbf{ThS. Bùi Trọng Hiếu} -- người đã tận tình định hướng, góp ý chuyên môn và tạo điều kiện thuận lợi để nhóm hoàn thành đề tài. Những nhận xét khoa học, sự nghiêm khắc trong phương pháp làm việc và tinh thần trách nhiệm của thầy là nền tảng quan trọng giúp chúng tôi hình thành tư duy nghiên cứu và hoàn thiện sản phẩm theo đúng chuẩn mực học thuật.

    Chúng tôi xin trân trọng cảm ơn \textbf{Khoa Công nghệ Thông tin -- Trường Đại học Giao thông vận tải TP.HCM} đã cung cấp môi trường học tập, tài liệu tham khảo và cơ sở hạ tầng cần thiết trong suốt quá trình học tập cũng như khi triển khai luận văn. Đồng thời, chúng tôi xin cảm ơn các thầy cô trong Khoa đã trang bị những kiến thức nền tảng về công nghệ phần mềm, cơ sở dữ liệu và trí tuệ nhân tạo -- là những mảnh ghép không thể thiếu để hiện thực hóa đề tài.

Cuối cùng, chúng tôi xin cảm ơn gia đình và bạn bè đã luôn đồng hành, khích lệ tinh thần và tạo điều kiện về thời gian để có thể tập trung hoàn thành luận văn. Mặc dù đã cố gắng hết sức, luận văn khó tránh khỏi những thiếu sót; chúng tôi kính mong nhận được sự góp ý của quý thầy cô để tiếp tục hoàn thiện trong học tập và công việc sau này.

\newpage

\chapter*{Tóm tắt luận văn}
\phantomsection
\addcontentsline{toc}{chapter}{Tóm tắt luận văn}
Trong bối cảnh chuyển đổi số giáo dục và sự phát triển mạnh mẽ của Trí tuệ nhân tạo tạo sinh, các hệ thống Quản lý học tập (LMS) truyền thống tại Việt Nam vẫn chủ yếu đóng vai trò lưu trữ học liệu, thiếu khả năng tương tác thông minh, chưa hỗ trợ hiệu quả nhu cầu học tập cá nhân hóa và còn hạn chế trong việc giám sát liêm chính học thuật khi tổ chức kiểm tra trực tuyến. Xuất phát từ yêu cầu thực tiễn đó, nhóm tác giả (\textbf{Nguyễn Quốc Tùng} và \textbf{Nguyễn Võ Thành}) nghiên cứu và xây dựng hệ thống \textbf{EduVerse} -- một nền tảng Smart LMS đa vai trò (Admin, Teacher, Student, Parent) với trọng tâm là mô-đun \textbf{AI Tutor} theo kiến trúc \textbf{Retrieval-Augmented Generation (RAG)} và mô-đun \textbf{giám sát thi/chống gian lận} dựa trên nhật ký sự kiện kết hợp AI tổng hợp.

Về công nghệ cốt lõi, hệ thống được hiện thực hóa bằng \textbf{Next.js App Router} theo mô hình full-stack, cho phép tổ chức giao diện và API trong cùng codebase đồng thời đảm bảo các tác vụ nhạy cảm được thực thi phía server. Dữ liệu nghiệp vụ được quản lý bằng \textbf{PostgreSQL} và \textbf{Prisma ORM}. Để phục vụ tìm kiếm ngữ nghĩa trong RAG, luận văn sử dụng \textbf{pgvector} để lưu trữ embedding và thực hiện truy vấn tương đồng Top-$k$ trực tiếp trên PostgreSQL. Về lớp AI, hệ thống sử dụng \textbf{Gemini API} với mô hình sinh nội dung \textbf{\texttt{gemini-2.5-flash-lite}} cho các tác vụ AI Tutor, tóm tắt chống gian lận và gợi ý chấm bài; đồng thời sử dụng mô hình embedding \texttt{gemini-embedding-001} để tạo vector ngữ nghĩa cho tài liệu/câu hỏi. Pipeline indexing triển khai cơ chế \textbf{contentHash} (SHA-256) và \textbf{upsert} theo khóa \texttt{(lessonId, chunkIndex)} nhằm đảm bảo tính lũy đẳng, giảm chi phí gọi embedding khi nội dung không đổi và tránh dữ liệu trùng.

Kết quả đạt được cho thấy EduVerse đáp ứng được các luồng nghiệp vụ quan trọng: học sinh có thể tương tác AI Tutor theo bài học và nhận câu trả lời có nguồn tham khảo từ học liệu; giáo viên theo dõi hành vi làm bài dựa trên event log và nhận tóm tắt chống gian lận; phụ huynh theo dõi tiến độ học tập; quản trị viên quản lý người dùng và cấu hình hệ thống. Ở kịch bản đo thực nghiệm end-to-end, độ trễ phản hồi trung bình của API RAG Tutor đạt 11.8s (P95 13.1s); cơ chế ghi nhận \texttt{ExamEvent} đạt 300/300 request thành công (0\% lỗi) với độ trễ trung bình 5.9s (P95 7.6s); và endpoint AI Anti-cheat Summary đạt độ trễ trung bình 10.5s (P95 12.1s).

\newpage
\tableofcontents 
\newpage
\chapter*{Danh mục từ viết tắt}
\phantomsection
\addcontentsline{toc}{chapter}{Danh mục từ viết tắt}
\begin{table}[H]
\centering
\small
\begin{tabularx}{\textwidth}{|p{3cm}|X|}
\hline
\textbf{Từ viết tắt} & \textbf{Giải thích} \\ \hline
LMS & Learning Management System -- Hệ thống quản lý học tập. \\ \hline
RAG & Retrieval-Augmented Generation -- Sinh nội dung tăng cường truy xuất tri thức từ kho tài liệu. \\ \hline
LLM & Large Language Model -- Mô hình ngôn ngữ lớn. \\ \hline
RBAC & Role-Based Access Control -- Kiểm soát truy cập theo vai trò người dùng. \\ \hline
JWT & JSON Web Token -- Token dạng JSON dùng để xác thực/ủy quyền. \\ \hline
ORM & Object-Relational Mapping -- Ánh xạ đối tượng--quan hệ. \\ \hline
API & Application Programming Interface -- Giao diện lập trình ứng dụng. \\ \hline
ACID & Atomicity, Consistency, Isolation, Durability -- Thuộc tính đảm bảo tính đúng đắn của giao dịch CSDL. \\ \hline
UI/UX & User Interface/User Experience -- Giao diện người dùng/Trải nghiệm người dùng. \\ \hline
\end{tabularx}
\end{table}

\newpage
\listoffigures
\newpage
\listoftables

% ======================================================
% CHƯƠNG 1: MỞ ĐẦU
% ======================================================
\newpage
\pagenumbering{arabic}
\chapter{MỞ ĐẦU}

\section{Lý do chọn đề tài}
Trong bối cảnh chuyển đổi số giáo dục, học tập trực tuyến và mô hình lớp học kết hợp (blended learning) ngày càng phổ biến. Vai trò của giáo viên vì thế cũng dịch chuyển từ người truyền thụ kiến thức sang người thiết kế hoạt động học tập, hướng dẫn và đánh giá. Tuy nhiên, khi quy mô lớp học tăng và hình thức học tập phân tán theo thời gian/không gian, giáo viên phải đối mặt với nhiều thách thức mới mà các hệ thống LMS truyền thống chưa giải quyết tốt.

Thứ nhất, \textbf{quá tải công việc của giáo viên} không chỉ đến từ giảng dạy mà còn từ khối lượng tác vụ lặp lại: tổ chức học liệu, giao bài và thu bài, trả lời câu hỏi thường gặp, theo dõi tiến độ từng học sinh, nhắc nhở deadline, và tổng hợp báo cáo. Trong môi trường trực tuyến, các tương tác này diễn ra thường xuyên hơn do học sinh ít có cơ hội hỏi trực tiếp. Nếu mọi câu hỏi đều phải chờ giáo viên phản hồi thủ công, hệ thống sẽ tạo ra ``nút thắt'' về thời gian, ảnh hưởng đến tốc độ học và trải nghiệm của học sinh.

Thứ hai, \textbf{nhu cầu hỗ trợ học tập 24/7 và cá nhân hóa} trở nên cấp thiết. Học sinh có mức độ tiếp thu khác nhau; cùng một bài giảng có thể phát sinh nhiều câu hỏi theo ngữ cảnh cá nhân (điểm chưa hiểu, ví dụ minh họa, cách áp dụng kiến thức, gợi ý làm bài). Trong khi đó, giáo viên khó có thể đồng thời hỗ trợ từng học sinh vào mọi thời điểm. Một hệ thống Smart LMS cần cung cấp cơ chế hỗ trợ ``đúng lúc'' (just-in-time) để học sinh tự giải quyết vướng mắc, đồng thời vẫn đảm bảo câu trả lời bám sát nội dung bài giảng và định hướng của giáo viên.

Thứ ba, \textbf{vấn đề gian lận thi cử trực tuyến ngày càng tinh vi}. Khi bài kiểm tra được thực hiện qua trình duyệt, học sinh có thể lợi dụng các hành vi như chuyển tab, thoát fullscreen, sao chép/dán nội dung, hoặc sử dụng thiết bị/phần mềm hỗ trợ ngoài luồng. Đối với giáo viên, việc rà soát thủ công nhật ký sự kiện (event log) theo từng học sinh/attempt là rất tốn thời gian, đặc biệt khi lớp học đông và bài kiểm tra diễn ra thường xuyên. Do đó, hệ thống cần có cơ chế định lượng mức độ nghi ngờ theo các quy tắc nhất quán, đồng thời cung cấp tóm tắt diễn giải để giáo viên nhanh chóng nắm bắt ``tín hiệu rủi ro'' mà không phải đọc toàn bộ log.

Từ các vấn đề trên, đề tài \textbf{nghiên cứu và xây dựng hệ thống quản lý học tập thông minh (Smart LMS) dựa trên kiến trúc RAG và mô hình ngôn ngữ lớn} được lựa chọn nhằm:
\begin{itemize}
    \item giảm tải công việc lặp lại cho giáo viên;
    \item hỗ trợ học sinh hỏi đáp theo ngữ cảnh học liệu với mức độ tin cậy cao;
    \item tăng cường liêm chính học thuật trong thi cử trực tuyến bằng giám sát dựa trên event log kết hợp AI tổng hợp.
\end{itemize}

\section{Mục tiêu nghiên cứu}
\begin{itemize}
    \item Thiết kế kiến trúc hệ thống LMS đa vai trò (Admin, Teacher, Student, Parent).
    \item Hiện thực hóa module AI Tutor hỗ trợ học tập dựa trên cơ sở dữ liệu vector.
    \item Xây dựng quy trình tự động hóa đánh giá năng lực học sinh qua AI.
    \item Đảm bảo an ninh và tính liêm chính học thuật qua hệ thống chống gian lận.
\end{itemize}

\section{Đối tượng và phạm vi nghiên cứu}
\begin{itemize}
    \item \textbf{Đối tượng:} Quy trình quản lý và hỗ trợ học tập tại các trường trung học.
    \item \textbf{Phạm vi:} Xây dựng ứng dụng Web dựa trên Next.js và tích hợp Gemini API; sử dụng \texttt{gemini-2.5-flash-lite} cho các tác vụ sinh nội dung và \texttt{gemini-embedding-001} cho tác vụ tạo embedding.
\end{itemize}

\section{Cấu trúc luận văn}
Nội dung luận văn được tổ chức thành 6 chương như sau:
\begin{itemize}
    \item \textbf{Chương 1 -- Mở đầu:} trình bày bối cảnh, lý do chọn đề tài, mục tiêu và phạm vi nghiên cứu.
    \item \textbf{Chương 2 -- Cơ sở lý thuyết và công nghệ:} tổng hợp các nền tảng kỹ thuật sử dụng trong hệ thống, bao gồm kiến trúc web, xác thực/phân quyền, cơ sở dữ liệu quan hệ--vector và dịch vụ AI.
    \item \textbf{Chương 3 -- Phân tích và thiết kế hệ thống:} mô tả tác nhân, yêu cầu chức năng/phi chức năng, thiết kế kiến trúc và đặc tả các Use Case trọng tâm.
    \item \textbf{Chương 4 -- Hiện thực hóa hệ thống:} trình bày cách hiện thực các mô-đun cốt lõi như AI Tutor theo kiến trúc RAG, cơ chế giám sát thi/chống gian lận và hỗ trợ chấm bài.
    \item \textbf{Chương 5 -- Đánh giá và thử nghiệm:} đánh giá hệ thống theo các kịch bản kiểm thử chức năng, độ trễ và mức độ hữu ích của các mô-đun AI.
    \item \textbf{Chương 6 -- Kết luận và hướng phát triển:} tổng kết kết quả đạt được, phân tích hạn chế và đề xuất hướng nâng cấp hệ thống trong tương lai.
\end{itemize}

% ======================================================
% CHƯƠNG 2: CƠ SỞ LÝ THUYẾT VÀ CÔNG NGHỆ
% ======================================================
\chapter{CƠ SỞ LÝ THUYẾT VÀ CÔNG NGHỆ}

\section{Tổng quan nền tảng công nghệ}
Hệ thống \textbf{EduVerse} được hiện thực hóa dưới dạng ứng dụng Web theo mô hình full-stack, trong đó lớp giao diện và lớp xử lý nghiệp vụ (API) được tổ chức thống nhất. Nhóm công nghệ trọng tâm gồm: Next.js (App Router) cho kiến trúc web hiện đại \cite{nextjs}, NextAuth.js cho xác thực và quản lý phiên đăng nhập \cite{nextauth}, Prisma ORM cho truy cập dữ liệu \cite{prisma}, PostgreSQL cho lưu trữ quan hệ \cite{postgresql}, pgvector cho truy xuất tương đồng vector phục vụ RAG \cite{pgvector}, Supabase Storage cho lưu trữ tệp nộp bài \cite{supabase}, và Gemini API cho các tác vụ AI (embedding và sinh nội dung) \cite{gemini}. Bảng \ref{tab:tech-stack} tóm tắt các công nghệ chính và vai trò tương ứng.

\begin{table}[htbp]
\centering
\small
\begingroup
\setlength{\tabcolsep}{4pt}
\renewcommand{\arraystretch}{1.15}
\begin{tabularx}{\textwidth}{|p{3.4cm}|>{\raggedright\arraybackslash}X|>{\raggedright\arraybackslash}X|}
\hline
\textbf{Công nghệ (phiên bản)} & \textbf{Vai trò trong hệ thống} & \textbf{Lý do lựa chọn} \\ \hline
Next.js App Router (14.2.33) \cite{nextjs} & Full-stack Web framework; routing, rendering, API Routes & Thống nhất UI và API trong cùng codebase; hỗ trợ triển khai logic nhạy cảm phía server; tối ưu hiệu năng và bảo mật \\ \hline
NextAuth.js (4.24.11) \cite{nextauth} & Xác thực người dùng, quản lý phiên đăng nhập (JWT), RBAC & Hỗ trợ đa cơ chế đăng nhập (Credentials, OAuth) và cho phép kiểm soát phiên/role nhất quán ở middleware và API \\ \hline
Prisma ORM (6.19.0) \cite{prisma} & Truy cập dữ liệu type-safe, migrations, ánh xạ mô hình dữ liệu & Giảm lỗi truy vấn, tăng tính nhất quán và hỗ trợ mở rộng schema có kiểm soát \\ \hline
PostgreSQL \cite{postgresql} & Cơ sở dữ liệu quan hệ lưu dữ liệu nghiệp vụ & Đảm bảo ACID, ổn định và phù hợp dữ liệu có quan hệ phức tạp (khóa học, bài tập, sự kiện thi) \\ \hline
pgvector \cite{pgvector} & Lưu embedding và truy vấn tương đồng vector phục vụ RAG & Cho phép truy xuất ngữ nghĩa ngay trên Postgres, thống nhất hạ tầng dữ liệu \\ \hline
Supabase Storage (2.76.1) \cite{supabase} & Lưu trữ tệp nộp bài, cấp signed URL & Tách tải lưu trữ file khỏi DB, hỗ trợ truy cập an toàn theo thời hạn \\ \hline
Gemini API  \cite{gemini} & Sinh nội dung cho AI Tutor/Anti-cheat summary/AI grading (model \texttt{gemini-2.5-flash-lite}); tạo embedding (\texttt{gemini-embedding-001}) & Cân bằng tốc độ và chi phí cho tác vụ sinh nội dung; thống nhất hệ sinh thái Gemini cho cả embedding và generation \\ \hline
SWR (2.3.6) & Fetching/caching dữ liệu phía client (revalidate) & Đơn giản hóa đồng bộ dữ liệu UI, giảm tải gọi lại API và cải thiện trải nghiệm người dùng \\ \hline
Vitest (4.0.15) & Kiểm thử tự động (unit test) & Tối ưu tốc độ chạy test; phù hợp dự án TypeScript/ESM và CI \\ \hline
lucide-react (0.546.0) & Bộ biểu tượng UI cho dashboard & Đồng bộ thiết kế icon, dễ mở rộng và tái sử dụng ở nhiều portal \\ \hline
\end{tabularx}
\endgroup
\caption{Tóm tắt Tech Stack sử dụng trong hệ thống EduVerse}
\label{tab:tech-stack}
\end{table}

\begin{table}[htbp]
\centering
\small
\begin{tabularx}{\textwidth}{|p{3cm}|X|X|}
\hline
\textbf{Tiêu chí} & \textbf{RAG \cite{rag}} & \textbf{LLM thuần} \\ \hline
Nguồn tri thức & Dựa trên tài liệu nội bộ (bài giảng) được truy xuất theo câu hỏi & Chủ yếu dựa trên tri thức đã học trong quá trình huấn luyện \\ \hline
Tính kiểm chứng & Có thể truy vết theo đoạn truy xuất; dễ giải thích tại sao trả lời như vậy & Khó truy vết nguồn; dễ sinh nội dung không có trong tài liệu \\ \hline
Rủi ro "ảo giác" & Thấp hơn do bị ràng buộc bởi ngữ cảnh truy xuất & Cao hơn, đặc biệt với câu hỏi đặc thù nội dung lớp học \\ \hline
Chi phí cập nhật tri thức & Cập nhật bằng cách thay tài liệu/embedding (không cần huấn luyện lại) & Thường cần fine-tune hoặc chờ model mới để cập nhật tri thức \\ \hline
Phù hợp giáo dục & Phù hợp khi cần bám sát học liệu và chuẩn hóa nội dung & Phù hợp cho gợi ý chung, nhưng cần kiểm soát chặt khi dùng như nguồn kiến thức \\ \hline
\end{tabularx}
\caption{So sánh RAG và LLM thuần trong bối cảnh hệ thống học tập}
\label{tab:rag-vs-llm}
\end{table}

\section{Kiến trúc tổng thể hệ thống}
Về mặt kiến trúc, EduVerse có thể xem như một hệ thống gồm bốn lớp chính:
\begin{itemize}
    \item \textbf{Lớp giao diện (Presentation):} các trang dashboard cho học sinh/giáo viên, tập trung vào thao tác học tập, làm bài và theo dõi kết quả.
    \item \textbf{Lớp dịch vụ ứng dụng (Application/API):} các API Routes xử lý nghiệp vụ (xác thực, nộp bài, ghi sự kiện thi, gọi AI Tutor, tóm tắt chống gian lận), đóng vai trò ranh giới bảo mật và kiểm soát truy cập.
    \item \textbf{Lớp dữ liệu (Data):} PostgreSQL lưu dữ liệu quan hệ, kết hợp pgvector để truy vấn vector; Supabase Storage lưu tệp đính kèm.
    \item \textbf{Lớp dịch vụ AI (External AI Service):} Gemini API đảm nhiệm tạo embedding (\texttt{gemini-embedding-001}) và sinh nội dung (\texttt{gemini-2.5-flash-lite}).
\end{itemize}
Trong bối cảnh Next.js, các API Routes và logic truy xuất dữ liệu được thực thi phía server, giúp đảm bảo các tác vụ nhạy cảm (khóa API, truy vấn dữ liệu) không bị lộ ra phía client \cite{nextjs}.

\section{Kiến trúc Web: Next.js App Router và React}
Next.js App Router cung cấp cơ chế định tuyến theo thư mục và hỗ trợ Server Components, giúp tách bạch phần render phía server và tương tác phía client. Trong EduVerse, API Routes được dùng để đóng gói các nghiệp vụ nhạy cảm (ví dụ gọi AI, truy cập cơ sở dữ liệu), qua đó hạn chế lộ khóa dịch vụ và giảm tải cho phía client \cite{nextjs}.

Ở cấp độ triển khai, App Router giúp thống nhất các luồng:
\begin{itemize}
    \item \textbf{SSR/Server Components:} phù hợp khi cần render dữ liệu theo người dùng (ví dụ thông tin lớp học, bài tập), giảm thời gian chờ nhờ trả về HTML đã có dữ liệu.
    \item \textbf{Client Components:} xử lý tương tác trực tiếp (làm bài kiểm tra, thao tác nộp bài, theo dõi trạng thái) và các hành vi cần bắt sự kiện theo thời gian thực.
    \item \textbf{API Routes:} cung cấp endpoint thống nhất cho client, đồng thời là nơi áp dụng các chính sách bảo mật (xác thực, phân quyền, giới hạn tần suất).
\end{itemize}

\section{Xác thực và phân quyền: NextAuth.js}
Hệ thống sử dụng NextAuth.js để hiện thực hóa đăng nhập và quản lý phiên làm việc của người dùng. Cách tiếp cận này cho phép tích hợp nhiều nhà cung cấp đăng nhập (ví dụ OAuth) và đồng thời hỗ trợ cơ chế Credentials khi cần. Dữ liệu phiên (session) được sử dụng để thực thi phân quyền theo vai trò (giáo viên, học sinh, quản trị, phụ huynh) trong các API và luồng điều hướng \cite{nextauth}.

Về mô hình triển khai, NextAuth.js cung cấp cơ chế quản lý phiên dựa trên JWT hoặc session lưu trong cơ sở dữ liệu. Khi người dùng đăng nhập, hệ thống tạo thông tin phiên để:
\begin{itemize}
    \item định danh người dùng (id/email);
    \item gắn \textbf{vai trò (role)} để kiểm soát truy cập chức năng;
    \item đảm bảo tính nhất quán khi truy cập các API Routes.
 \end{itemize}

\subsection{Phân quyền theo vai trò (RBAC) và bảo vệ truy cập}
EduVerse áp dụng RBAC (Role-Based Access Control) với bốn vai trò \texttt{STUDENT}, \texttt{TEACHER}, \texttt{PARENT}, \texttt{ADMIN}. Trong EduVerse, RBAC không chỉ được dùng để kiểm tra quyền ở API mà còn được hiện thực hóa thành mô hình \textbf{multi-portal trong cùng một ứng dụng Next.js}: mỗi vai trò có một không gian chức năng riêng (ví dụ \texttt{/dashboard/student}, \texttt{/dashboard/teacher}, \texttt{/dashboard/parent}, \texttt{/dashboard/admin}) nhưng dùng chung một cơ chế xác thực và hạ tầng dữ liệu.

Để hạn chế truy cập sai vai trò và tăng tính an toàn khi vận hành, hệ thống áp dụng cách tiếp cận \textbf{phòng thủ nhiều lớp}:
\begin{itemize}
    \item \textbf{Tầng phiên (NextAuth session/JWT):} đồng bộ \texttt{role} và \texttt{roleSelectedAt} vào token nhằm đảm bảo quyết định phân quyền có thể tái sử dụng xuyên suốt UI, middleware và API.
    \item \textbf{Tầng middleware:} kiểm tra token hợp lệ, điều hướng người dùng về đúng portal theo role và chặn truy cập cross-role.
    \item \textbf{Tầng API Routes:} mỗi endpoint tiếp tục kiểm tra role và quyền sở hữu dữ liệu (ownership) trước khi truy vấn CSDL hoặc gọi dịch vụ AI.
\end{itemize}

\subsection{Cơ chế JWT (JSON Web Token) và xác thực ở Middleware}
JWT là token dạng \texttt{header.payload.signature}. \texttt{payload} chứa các claim (ví dụ \texttt{id}, \texttt{email}, \texttt{role}, \texttt{roleSelectedAt}), còn \texttt{signature} là chữ ký số được tạo từ \texttt{header + payload} bằng một khóa bí mật. Nhờ chữ ký, hệ thống có thể phát hiện token bị sửa đổi trái phép.

Trong mô hình triển khai của EduVerse, NextAuth sử dụng chiến lược JWT để giảm phụ thuộc vào session store và giúp middleware ra quyết định nhanh. Khi request đi qua middleware, hệ thống đọc token và xác thực chữ ký (dựa trên \texttt{NEXTAUTH\_SECRET}); nếu token không hợp lệ/hết hạn, request bị chuyển hướng về trang đăng nhập. Nếu hợp lệ, middleware sử dụng các claim để:
\begin{itemize}
    \item bắt buộc người dùng hoàn tất chọn vai trò trước khi vào portal (dựa trên \texttt{roleSelectedAt});
    \item điều hướng đúng dashboard theo role;
    \item từ chối truy cập các tuyến/endpoint không thuộc quyền.
\end{itemize}

\section{Tầng dữ liệu: Prisma ORM và PostgreSQL}
Prisma ORM cung cấp lớp truy cập dữ liệu type-safe cho TypeScript, giảm lỗi truy vấn và hỗ trợ migration có kiểm soát \cite{prisma}. PostgreSQL được lựa chọn cho dữ liệu nghiệp vụ nhờ đặc tính ACID, cơ chế ràng buộc khóa ngoại và khả năng mở rộng cho dữ liệu quan hệ phức tạp (lớp học, bài tập, attempt, sự kiện thi) \cite{postgresql}.

\section{Cơ sở dữ liệu vector: pgvector}
Đối với AI Tutor theo kiến trúc RAG, hệ thống cần truy vấn ngữ nghĩa theo vector embedding. EduVerse sử dụng pgvector để lưu vector và thực hiện similarity search trực tiếp trong PostgreSQL \cite{pgvector}. Cách tiếp cận này giúp thống nhất hạ tầng dữ liệu (relational + vector) và cho phép kết hợp điều kiện lọc nghiệp vụ (theo \texttt{lessonId}, \texttt{courseId}) trong cùng truy vấn.

\section{Lưu trữ tệp: Supabase Storage}
Các tệp đính kèm bài giảng và bài nộp dạng file được lưu trữ tại Supabase Storage, tách tải file khỏi CSDL quan hệ. Hệ thống sử dụng cơ chế signed URL để cấp quyền truy cập theo thời hạn, giảm rủi ro lộ đường dẫn tĩnh \cite{supabase}.

\section{Dịch vụ AI: Gemini API}
EduVerse sử dụng Gemini API cho hai nhóm tác vụ chính \cite{gemini}: (i) \textbf{tạo embedding} bằng \texttt{gemini-embedding-001} phục vụ lập chỉ mục và truy vấn ngữ nghĩa, và (ii) \textbf{sinh nội dung} bằng \texttt{gemini-2.5-flash-lite} cho AI Tutor, tóm tắt chống gian lận và gợi ý chấm bài. Việc lựa chọn phiên bản \textit{flash-lite} hướng tới cân bằng giữa độ trễ, chi phí và chất lượng phản hồi trong bối cảnh hệ thống có nhiều request tương tác.

\section{Kiến trúc RAG (Retrieval-Augmented Generation)}
RAG kết hợp truy xuất tri thức và sinh câu trả lời: hệ thống tạo embedding cho câu hỏi, truy vấn Top-$k$ đoạn học liệu liên quan, sau đó đưa ngữ cảnh truy xuất vào prompt để mô hình sinh câu trả lời bám sát tài liệu. So với LLM thuần, RAG giúp tăng khả năng kiểm chứng và giảm hiện tượng "ảo giác" khi câu hỏi gắn với nội dung lớp học \cite{rag}.

\section{Tổng quan chống gian lận dựa trên event log}
Mô-đun giám sát thi của EduVerse thu thập sự kiện (event log) trong quá trình làm bài (ví dụ chuyển tab, thoát fullscreen, thao tác clipboard). Các sự kiện được chuẩn hóa và tổng hợp để tính \texttt{suspicionScore} theo quy tắc; sau đó \texttt{gemini-2.5-flash-lite} có thể tạo tóm tắt diễn giải nhằm hỗ trợ giáo viên rà soát nhanh các attempt có rủi ro cao.

\chapter{PHÂN TÍCH VÀ THIẾT KẾ HỆ THỐNG}

\section{Giới thiệu chương}
Chương 3 trình bày phân tích và thiết kế hệ thống \textbf{EduVerse} dựa trên hiện thực triển khai. Nội dung tập trung làm rõ đối tượng sử dụng, yêu cầu chức năng/phi chức năng và thiết kế các thành phần cốt lõi. Trọng tâm thiết kế được nhấn mạnh ở hai mô-đun: \textbf{AI Tutor theo kiến trúc RAG} và \textbf{giám sát thi/chống gian lận dựa trên event log kết hợp AI tổng hợp}.

\section{Phân tích hệ thống}
\subsection{Đối tượng sử dụng (Actor) và vai trò (Role)}
EduVerse phục vụ các tác nhân chính:
\begin{itemize}
    \item \textbf{Học sinh (STUDENT):} tham gia lớp, học bài, làm bài tập/kiểm tra, tương tác AI Tutor.
    \item \textbf{Giáo viên (TEACHER):} tạo lớp, quản lý khóa học/bài học, tạo bài tập (Quiz/Essay), theo dõi tiến độ và giám sát thi.
    \item \textbf{Phụ huynh (PARENT):} liên kết với học sinh, theo dõi tiến độ/kết quả và nhận tóm tắt định kỳ.
    \item \textbf{Quản trị viên (ADMIN):} quản trị người dùng và cấu hình hệ thống; theo dõi nhật ký hoạt động.
\end{itemize}

\subsection{Yêu cầu chức năng}
Các yêu cầu chức năng được mô tả theo tác nhân sử dụng chính như sau:
\begin{itemize}
    \item \textbf{Khách (chưa đăng nhập):} đăng ký, đăng nhập, quên/đặt lại mật khẩu.
    \item \textbf{Học sinh:} tham gia lớp; xem bài học và học liệu; làm bài (quiz/essay); nộp bài; nhận thông báo; nhắn tin; tương tác AI Tutor.
    \item \textbf{Giáo viên:} tạo và quản lý lớp học; quản lý khóa học/bài học; tạo bài tập/đề kiểm tra; theo dõi tiến độ và kết quả; giám sát thi dựa trên event log và nhận hỗ trợ tổng hợp bằng AI.
    \item \textbf{Phụ huynh:} liên kết với học sinh; theo dõi tiến độ/kết quả; trao đổi với giáo viên.
    \item \textbf{Quản trị viên:} quản trị người dùng và cấu hình hệ thống; theo dõi nhật ký hoạt động.
\end{itemize}

\subsection{Yêu cầu phi chức năng}
\begin{itemize}
    \item \textbf{Bảo mật và phân quyền:} hệ thống dùng NextAuth (JWT strategy) và middleware điều hướng/phân quyền theo role; các API nhạy cảm giới hạn theo vai trò (ví dụ: teacher-only, admin-only) \cite{nextauth}.
    \item \textbf{Kiểm soát lạm dụng (rate limit):} các API nhạy cảm (reset password, AI Tutor, AI anti-cheat, cập nhật cấu hình hệ thống) áp dụng rate limit theo IP và/hoặc user.
    \item \textbf{Khả năng truy vết:} hệ thống ghi audit log cho các hành vi quan trọng và lưu event log cho giám sát thi.
\end{itemize}

\section{Thiết kế hệ thống}
\subsection{Thiết kế kiến trúc}
EduVerse triển khai theo mô hình \textbf{Client--Server} trong một codebase Next.js: client UI gửi request đến API Routes; API Routes thực thi nghiệp vụ, truy cập dữ liệu qua Prisma và gọi dịch vụ AI.

\begin{figure}[H]
    \centering
    \IfFileExists{docs/luanvan/figures/uml/EduVerse_Component_Architecture.png}{
        \includegraphics[width=0.95\textwidth]{docs/luanvan/figures/uml/EduVerse_Component_Architecture.png}
    }{
        \IfFileExists{figures/uml/EduVerse_Component_Architecture.png}{
            \includegraphics[width=0.95\textwidth]{figures/uml/EduVerse_Component_Architecture.png}
        }{
            \IfFileExists{../../out/docs/luanvan/figures/uml/EduVerse_Component_Architecture.png}{
                \includegraphics[width=0.95\textwidth]{../../out/docs/luanvan/figures/uml/EduVerse_Component_Architecture.png}
            }{
                \IfFileExists{../out/docs/luanvan/figures/uml/EduVerse_Component_Architecture.png}{
                    \includegraphics[width=0.95\textwidth]{../out/docs/luanvan/figures/uml/EduVerse_Component_Architecture.png}
                }{
                    \IfFileExists{out/docs/luanvan/figures/uml/EduVerse_Component_Architecture.png}{
                        \includegraphics[width=0.95\textwidth]{out/docs/luanvan/figures/uml/EduVerse_Component_Architecture.png}
                    }{
                        \framebox{\parbox{0.95\textwidth}{\centering\vspace{1cm}Component/Architecture Diagram (PlantUML PNG)\vspace{1cm}}}
                    }
                }
            }
        }
    }
    \caption{Kiến trúc tổng thể hệ thống EduVerse}
    \label{fig:component-architecture}
\end{figure}

\subsection{Thiết kế Use Case}
\Needspace{0.9\textheight}
\subsubsection{Use Case tổng thể hệ thống}
\begin{figure}[H]
    \centering
    \IfFileExists{../out/docs/luanvan/figures/uml/usecase-overall/usecase-overall.png}{
        \includegraphics[width=0.95\textwidth,height=0.85\textheight,keepaspectratio]{../out/docs/luanvan/figures/uml/usecase-overall/usecase-overall.png}
    }{
        \IfFileExists{out/docs/luanvan/figures/uml/usecase-overall/usecase-overall.png}{
            \includegraphics[width=0.95\textwidth,height=0.85\textheight,keepaspectratio]{out/docs/luanvan/figures/uml/usecase-overall/usecase-overall.png}
        }{
            \IfFileExists{docs/luanvan/figures/uml/usecase-overall.png}{
                \includegraphics[width=0.95\textwidth,height=0.85\textheight,keepaspectratio]{docs/luanvan/figures/uml/usecase-overall.png}
            }{
                \IfFileExists{figures/uml/usecase-overall.png}{
                    \includegraphics[width=0.95\textwidth,height=0.85\textheight,keepaspectratio]{figures/uml/usecase-overall.png}
                }{
                    \framebox{\parbox{0.95\textwidth}{\centering\vspace{1cm}Use Case Diagram (PlantUML PNG)\vspace{1cm}}}
                }
            }
        }
    }
    \caption{Use Case tổng thể hệ thống EduVerse}
    \label{fig:usecase-overall}
\end{figure}

\Needspace{0.9\textheight}
\subsubsection{Use Case module giám sát thi và chống gian lận}
\begin{figure}[H]
    \centering
    \IfFileExists{../out/docs/luanvan/figures/uml/usecase-anti-cheat/usecase-anti-cheat.png}{
        \includegraphics[width=0.95\textwidth,height=0.85\textheight,keepaspectratio]{../out/docs/luanvan/figures/uml/usecase-anti-cheat/usecase-anti-cheat.png}
    }{
        \IfFileExists{out/docs/luanvan/figures/uml/usecase-anti-cheat/usecase-anti-cheat.png}{
            \includegraphics[width=0.95\textwidth,height=0.85\textheight,keepaspectratio]{out/docs/luanvan/figures/uml/usecase-anti-cheat/usecase-anti-cheat.png}
        }{
            \IfFileExists{docs/luanvan/figures/uml/usecase-anti-cheat.png}{
                \includegraphics[width=0.95\textwidth,height=0.85\textheight,keepaspectratio]{docs/luanvan/figures/uml/usecase-anti-cheat.png}
            }{
                \IfFileExists{figures/uml/usecase-anti-cheat.png}{
                    \includegraphics[width=0.95\textwidth,height=0.85\textheight,keepaspectratio]{figures/uml/usecase-anti-cheat.png}
                }{
                    \framebox{\parbox{0.95\textwidth}{\centering\vspace{1cm}Use Case Diagram (PlantUML PNG)\vspace{1cm}}}
                }
            }
        }
    }
    \caption{Use Case tổng quan module giám sát thi và chống gian lận}
    \label{fig:usecase-anti-cheat}
\end{figure}

\begin{figure}[H]
    \centering
    \IfFileExists{figures/uml/usecase-teacher.png}{
        \includegraphics[width=0.95\textwidth,height=0.85\textheight,keepaspectratio]{figures/uml/usecase-teacher.png}
    }{
        \IfFileExists{docs/luanvan/figures/uml/usecase-teacher.png}{
            \includegraphics[width=0.95\textwidth,height=0.85\textheight,keepaspectratio]{docs/luanvan/figures/uml/usecase-teacher.png}
        }{
            \framebox{\parbox{0.95\textwidth}{\centering\vspace{1cm}Use Case Diagram (PlantUML PNG)\vspace{1cm}}}
        }
    }
    \caption{Use Case chi tiết cho vai trò Giáo viên trong EduVerse}
    \label{fig:usecase-teacher}
\end{figure}

\begin{table}[H]
\centering
\small
\begin{tabularx}{\textwidth}{|p{3.4cm}|X|}
\hline
\textbf{Tác nhân} & Học sinh (STUDENT) \\ \hline
\textbf{Tiền điều kiện} & (1) Học sinh đã đăng nhập và thuộc lớp. (2) Bài học tồn tại. (3) Lesson đã được lập chỉ mục embedding. \\ \hline
\textbf{Luồng sự kiện chính} & \textbf{B1.} Mở giao diện AI Tutor. \newline \textbf{B2.} Nhập câu hỏi. \newline \textbf{B3.} Hệ thống xác thực và kiểm tra giới hạn tần suất. \newline \textbf{B4.} Tạo embedding và truy vấn Top-$k$ đoạn học liệu liên quan. \newline \textbf{B5.} Gọi LLM sinh phản hồi dựa trên ngữ cảnh bài giảng. \newline \textbf{B6.} Trả về câu trả lời kèm nguồn tham khảo. \\ \hline
\textbf{Luồng ngoại lệ} & \textbf{E1.} Vượt quá lượt hỏi: trả lỗi 429. \newline \textbf{E2.} Chưa có dữ liệu embedding: trả thông báo hướng dẫn. \\ \hline
\end{tabularx}
\caption{Đặc tả Use Case: Học sinh tương tác với AI Tutor}
\label{tab:uc-student-ai-tutor}
\end{table}

\begin{table}[H]
\centering
\small
\begin{tabularx}{\textwidth}{|p{3.4cm}|X|}
\hline
\textbf{Tác nhân} & Giáo viên (TEACHER) \\ \hline
\textbf{Tiền điều kiện} & (1) Giáo viên đã đăng nhập. (2) Giáo viên là chủ sở hữu assignment/quiz. (3) Có dữ liệu \texttt{exam\_events} cho attempt cần xem. \\ \hline
\textbf{Luồng sự kiện chính} & \textbf{B1.} Mở trang báo cáo giám sát thi của quiz.\newline \textbf{B2.} Tải danh sách attempt và thống kê số lượng sự kiện.\newline \textbf{B3.} Chuẩn hóa \texttt{eventType} và tính \texttt{suspicionScore} theo rule-based.\newline \textbf{B4.} Phân loại \texttt{riskLevel} và hiển thị breakdown theo từng rule.\newline \textbf{B5.} Xem chi tiết timeline sự kiện theo attempt. \\ \hline
\textbf{Luồng ngoại lệ} & \textbf{E1.} Không phải chủ sở hữu assignment: trả HTTP 403.\newline \textbf{E2.} Không có event log: hiển thị empty state và vẫn trả dữ liệu nhất quán. \\ \hline
\textbf{Hậu điều kiện} & Giáo viên có căn cứ định lượng (điểm) và định tính (timeline) để đánh giá liêm chính học thuật. \\ \hline
\end{tabularx}
\caption{Đặc tả Use Case: Giáo viên xem báo cáo Anti-cheat}
\label{tab:uc-teacher-anti-cheat}
\end{table}

\subsubsection{Sơ đồ hoạt động (Activity Diagram): Tính điểm nghi ngờ Anti-cheat}
\begin{lstlisting}[caption={Mermaid Activity Diagram: Luồng tính suspicionScore Anti-cheat},label={lst:activity-anti-cheat-mermaid}]
flowchart TD
    A([Nhận danh sách exam events thô]) --> B{Validate input}
    B -- invalid --> X([Trả lỗi / bỏ qua event không hợp lệ])
    B -- valid --> C[Chuẩn hóa eventType\nTAB_SWITCH_DETECTED -> TAB_SWITCH\nCOPY_PASTE_ATTEMPT -> CLIPBOARD]
    C --> D[Đếm số lần theo từng loại sự kiện\ncountsByType[type]++]
    D --> E[Áp dụng trọng số theo quy tắc\npoints = min(maxPoints, count * pointsPerHit)]
    E --> F[Tổng hợp điểm toàn bộ quy tắc]
    F --> G[Giới hạn điểm về khoảng [0..100]]
    G --> H[Phân loại mức độ rủi ro\nLOW / MEDIUM / HIGH]
    H --> I([Trả về suspicionScore + breakdown + riskLevel])
    X --> I
\end{lstlisting}

\subsubsection{Sơ đồ tuần tự (Sequence Diagram): Quy trình RAG xử lý câu hỏi}
\begin{lstlisting}[caption={Mermaid Sequence Diagram: Quy trình RAG xử lý câu hỏi},label={lst:seq-rag-mermaid}]
sequenceDiagram
    participant Client as Student Client (Web)
    participant Server as Next.js API Route
    participant DB as PostgreSQL + pgvector
    participant Gemini as Gemini API

    Client->>Server: POST /api/ai/tutor/chat (lessonId, question)
    Server->>Server: Auth + membership + rate-limit
    Server->>Gemini: Embed(question, task=RETRIEVAL_QUERY)
    Gemini-->>Server: queryVector (1536 dims)
    Server->>DB: Similarity search TopK chunks
    DB-->>Server: chunks + metadata
    Server->>Gemini: Generate answer with retrieved context
    Gemini-->>Server: answer
    Server-->>Client: answer + sources
\end{lstlisting}

\begin{figure}[H]
    \centering
    \IfFileExists{figures/uml/erd-core-flow.png}{
        \includegraphics[width=0.95\textwidth,height=0.85\textheight,keepaspectratio]{figures/uml/erd-core-flow.png}
    }{
        \IfFileExists{docs/luanvan/figures/uml/erd-core-flow.png}{
            \includegraphics[width=0.95\textwidth,height=0.85\textheight,keepaspectratio]{docs/luanvan/figures/uml/erd-core-flow.png}
        }{
            \framebox{\parbox{0.95\textwidth}{\centering\vspace{1cm}ERD Diagram (PlantUML PNG)\vspace{1cm}}}
        }
    }
    \caption{ERD rút gọn mô tả luồng dữ liệu từ User đến Embedding}
    \label{fig:erd-core-flow}
\end{figure}

\subsection{Thiết kế cơ sở dữ liệu}
Hệ thống sử dụng Prisma ORM kết nối PostgreSQL \cite{prisma,postgresql}. Hình \ref{fig:db-core} minh họa mối quan hệ giữa các thực thể dữ liệu cốt lõi; Bảng \ref{tab:db-core} tóm tắt các nhóm dữ liệu và vai trò tương ứng.

\begin{figure}[H]
    \centering
    \IfFileExists{docs/luanvan/figures/uml/EduVerse_DB_Core.png}{
        \includegraphics[height=0.9\textheight,keepaspectratio]{docs/luanvan/figures/uml/EduVerse_DB_Core.png}
    }{
        \IfFileExists{figures/uml/EduVerse_DB_Core.png}{
            \includegraphics[height=0.9\textheight,keepaspectratio]{figures/uml/EduVerse_DB_Core.png}
        }{
            \IfFileExists{../../out/docs/luanvan/figures/uml/db-core/EduVerse_DB_Core.png}{
                \includegraphics[height=0.9\textheight,keepaspectratio]{../../out/docs/luanvan/figures/uml/db-core/EduVerse_DB_Core.png}
            }{
                \IfFileExists{../out/docs/luanvan/figures/uml/db-core/EduVerse_DB_Core.png}{
                    \includegraphics[height=0.9\textheight,keepaspectratio]{../out/docs/luanvan/figures/uml/db-core/EduVerse_DB_Core.png}
                }{
                    \IfFileExists{out/docs/luanvan/figures/uml/db-core/EduVerse_DB_Core.png}{
                        \includegraphics[height=0.9\textheight,keepaspectratio]{out/docs/luanvan/figures/uml/db-core/EduVerse_DB_Core.png}
                    }{
                        \framebox{\parbox{0.95\textwidth}{\centering\vspace{1cm}Database Diagram (PlantUML PNG)\vspace{1cm}}}
                    }
                }
            }
        }
    }
    \caption{Sơ đồ cơ sở dữ liệu cốt lõi của hệ thống EduVerse}
    \label{fig:db-core}
\end{figure}

\begin{table}[H]
\centering
\small
\begin{tabularx}{\textwidth}{|p{3.2cm}|X|}
\hline
\textbf{Nhóm dữ liệu} & \textbf{Vai trò} \\ \hline
Người dùng & Các bảng: \texttt{users} (\texttt{User}). Lưu thông tin tài khoản, vai trò (\texttt{UserRole}), thời điểm chọn vai trò (\texttt{roleSelectedAt}). \\ \hline
Khôi phục mật khẩu & Các bảng: \texttt{password\_resets} (\texttt{PasswordReset}). Lưu mã xác nhận đặt lại mật khẩu, thời hạn (\texttt{expires}) và trạng thái hoàn thành (\texttt{completed}). \\ \hline
Lớp học & Các bảng: \texttt{classrooms}, \texttt{classroom\_students}. Tổ chức lớp học, liên kết học sinh--lớp. \\ \hline
Khóa học/bài học & Các bảng: \texttt{courses}, \texttt{lessons}, \texttt{lesson\_attachments}. Lưu nội dung bài học và tệp đính kèm; là nguồn tri thức cho RAG Tutor. \\ \hline
RAG embedding & Bảng: \texttt{lesson\_embedding\_chunks}. Lưu các đoạn (chunk) đã embedding bằng pgvector để truy vấn tương đồng Top-$k$ \cite{pgvector}. \\ \hline
Bài tập/quiz & Các bảng: \texttt{assignments}, \texttt{questions}, \texttt{question\_options}. Lưu bài tập dạng ESSAY/QUIZ, cấu trúc câu hỏi và đáp án. \\ \hline
Nộp bài (ESSAY) & Bảng: \texttt{assignment\_submissions}. Lưu bài tự luận dạng nội dung, điểm và phản hồi. \\ \hline
Nộp bài (tệp) & Các bảng: \texttt{submissions}, \texttt{submission\_files}. Lưu metadata tệp nộp bài (\texttt{storagePath}, \texttt{mimeType}, \texttt{sizeBytes}). \\ \hline
Giám sát thi & Các bảng: \texttt{exam\_events}, \texttt{assignment\_attempts}. Lưu nhật ký sự kiện thi (event log) và thông tin attempt. \\ \hline
Chat & Các bảng: \texttt{conversations}, \texttt{conversation\_participants}, \texttt{messages}, \texttt{chat\_attachments}. Lưu hội thoại và tin nhắn. \\ \hline
Hệ thống/Audit & Các bảng: \texttt{system\_settings}, \texttt{audit\_logs}. Lưu cấu hình hệ thống và nhật ký hoạt động phục vụ truy vết. \\ \hline
\end{tabularx}
\caption{Các bảng dữ liệu cốt lõi trong hệ thống EduVerse }
\label{tab:db-core}
\end{table}

\subsubsection{Mô hình lớp học -- khóa học -- bài giảng và các quan hệ trung gian}
Trong thiết kế EduVerse, \textbf{Classroom} là đơn vị tổ chức lớp học do giáo viên quản lý (liên kết bởi \texttt{teacherId}), còn \textbf{Course} đóng vai trò như một gói nội dung học tập có thể tái sử dụng cho nhiều lớp khác nhau. Vì vậy, quan hệ giữa Classroom và Course không được ràng buộc 1--n trực tiếp, mà được hiện thực bằng bảng trung gian \textbf{ClassroomCourse}. Thiết kế này phản ánh đúng yêu cầu nghiệp vụ: một lớp có thể học nhiều khóa, và một khóa có thể được gắn cho nhiều lớp.

Trên nền đó, \textbf{Lesson} được mô hình hóa theo quan hệ 1--n từ Course: mỗi bài giảng bắt buộc thuộc về một Course thông qua \texttt{courseId} và có thứ tự hiển thị \texttt{order}. Quan hệ này sử dụng ràng buộc xóa lan truyền (\texttt{onDelete: Cascade}), đảm bảo khi xóa Course thì toàn bộ Lesson liên quan cũng được xóa để tránh dữ liệu mồ côi (\textit{orphan records}). Ngoài ra, Lesson có thể đính kèm tệp qua bảng \textbf{LessonAttachment} với chỉ mục theo \texttt{(lessonId, createdAt)} nhằm tối ưu truy vấn danh sách tệp theo thời gian.

\subsubsection{Thiết kế bài tập và hỗ trợ đa dạng hình thức ESSAY/QUIZ}
Bảng \textbf{Assignment} được thiết kế như một thực thể trung tâm cho cả hai dạng bài: \texttt{ESSAY} và \texttt{QUIZ} thông qua thuộc tính \texttt{type} (\texttt{AssignmentType}). Đây là một quyết định thiết kế theo hướng \textit{đa hình} (polymorphism) giúp mở rộng dễ dàng mà vẫn giữ cấu trúc dữ liệu nhất quán.

\begin{itemize}
    \item Với bài \textbf{ESSAY}: nội dung làm bài có thể được lưu trong \textbf{AssignmentSubmission} (trường \texttt{content}) cùng thời điểm nộp (\texttt{submittedAt}), điểm số (\texttt{grade}) và phản hồi (\texttt{feedback}). Thiết kế \texttt{@@unique(assignmentId, studentId, attempt)} đảm bảo mỗi học sinh chỉ có một bản nộp tương ứng với một lần làm (attempt), tránh ghi đè không kiểm soát.
    \item Với bài \textbf{QUIZ}: hệ thống tái sử dụng Assignment như một ``container'', còn cấu trúc đề thi được mô hình hóa bởi \textbf{Question} và \textbf{Option}. Mỗi Question thuộc về một Assignment (quan hệ 1--n) và có \texttt{type} (SINGLE/MULTIPLE/TRUE\_FALSE/FILL\_BLANK/ESSAY), cho phép đa dạng hóa hình thức câu hỏi. Mỗi Option thuộc về một Question (quan hệ 1--n), có \texttt{isCorrect} để đánh dấu đáp án đúng và \texttt{order} để kiểm soát thứ tự hiển thị.
\end{itemize}

Bên cạnh đó, Assignment được gắn với Classroom thông qua bảng trung gian \textbf{AssignmentClassroom} (ràng buộc \texttt{@@unique(classroomId, assignmentId)}). Cách tiếp cận này phù hợp với thực tế giảng dạy: cùng một bài tập có thể được giao cho nhiều lớp, đồng thời hỗ trợ thống kê theo lớp mà không làm thay đổi bản chất của Assignment.

\subsubsection{Bảng lesson\_embedding\_chunks và vai trò trong tìm kiếm vector (RAG)}
Để phục vụ truy xuất ngữ nghĩa cho AI Tutor, EduVerse lưu trữ embedding dưới dạng vector trong PostgreSQL thông qua phần mở rộng pgvector \cite{pgvector}. Bảng \textbf{lesson\_embedding\_chunks} lưu nội dung từng đoạn (chunk) sau khi chia nhỏ bài giảng, kèm \texttt{chunkIndex} ổn định theo từng lesson, và \texttt{contentHash} (SHA-256) nhằm phát hiện chunk không đổi để \textbf{bỏ qua indexing} khi nội dung giữ nguyên, từ đó giảm chi phí gọi embedding.

Trong truy vấn RAG, hệ thống tạo embedding cho câu hỏi của học sinh và thực hiện tìm kiếm Top-$k$ chunk gần nhất theo khoảng cách vector (ví dụ \texttt{("embedding" <=> queryVector::vector) AS distance}), sau đó sắp xếp \texttt{ORDER BY distance ASC} và lấy \texttt{LIMIT topK}. Kết quả là danh sách các đoạn nguồn được dùng để ràng buộc câu trả lời của AI Tutor, nhằm giảm hiện tượng “ảo giác” \cite{rag}.

\subsubsection{Dữ liệu giám sát thi và nguyên tắc thiết kế an toàn}
Đối với phân hệ giám sát thi, bảng \textbf{exam\_events} ghi nhận các sự kiện trong quá trình học sinh làm bài (ví dụ thoát fullscreen, chuyển tab, thao tác clipboard). Dữ liệu được lưu theo \texttt{assignmentId}, \texttt{studentId} và \texttt{attempt}. Trường \texttt{metadata} có kiểu JSON nhưng được giới hạn kích thước payload ở tầng API nhằm giảm rủi ro lạm dụng và đảm bảo hiệu năng lưu trữ. Các chỉ mục theo thời gian (ví dụ \texttt{(assignmentId, createdAt)} và \texttt{(studentId, createdAt)}) hỗ trợ truy vấn phục vụ dashboard giáo viên (lọc theo khoảng thời gian, sắp xếp mới nhất, ...).

\subsubsection{Tối ưu hiệu năng truy vấn bằng chỉ mục (Indexing)}
Ngoài việc mô hình hóa quan hệ, EduVerse triển khai chiến lược đánh chỉ mục (indexing) nhằm tối ưu các truy vấn có tần suất cao ở tầng PostgreSQL. Trên bảng \textbf{User}, chỉ mục trên \texttt{email} hỗ trợ truy vấn định danh (đăng nhập, tìm kiếm theo email), đồng thời chỉ mục kết hợp \texttt{(role, createdAt)} giúp tăng tốc các truy vấn quản trị như lọc theo vai trò và sắp xếp theo thời gian tạo. Trên bảng \textbf{Classroom}, chỉ mục theo \texttt{teacherId} và chỉ mục kết hợp \texttt{(teacherId, isActive)} phục vụ dashboard giáo viên khi liệt kê lớp đang hoạt động. Với phân hệ giám sát thi, bảng \textbf{ExamEvent} sử dụng các chỉ mục theo thời gian (ví dụ \texttt{(assignmentId, createdAt)} và \texttt{(studentId, createdAt)}) để hỗ trợ truy vấn theo bài thi/học sinh và sắp xếp các sự kiện mới nhất trong luồng theo dõi.

\subsubsection{Đảm bảo tính nguyên tử bằng giao dịch (Database Transactions)}
Trong các nghiệp vụ nhiều bước, hệ thống sử dụng \texttt{Prisma.\$transaction} để đảm bảo \textbf{Atomicity} (tính nguyên tử): toàn bộ thay đổi hoặc được ghi thành công, hoặc được rollback hoàn toàn khi có lỗi. Ví dụ, khi tạo user và gắn membership vào tổ chức, việc tạo bản ghi user và upsert bản ghi membership được đặt trong cùng một transaction nhằm tránh trạng thái ``tạo user nhưng thiếu quan hệ tổ chức''. Tương tự, các nghiệp vụ liên kết phụ huynh--học sinh (chấp nhận lời mời, duyệt yêu cầu liên kết) cập nhật trạng thái lời mời/yêu cầu và tạo bản ghi liên kết trong cùng một transaction, giảm rủi ro phát sinh trạng thái trung gian không hợp lệ.

\subsubsection{Sử dụng truy vấn thô (Raw SQL) cho bài toán đếm và thống kê}
Mặc dù Prisma cung cấp truy vấn type-safe, một số bài toán đếm (count) và tổng hợp (aggregation) phức tạp được hiện thực bằng \textbf{Raw SQL} để tối ưu hiệu năng và giảm độ phức tạp ở tầng ORM. Trong mô-đun Chat, hệ thống sử dụng truy vấn thô để tính \texttt{unread count} (đếm tin nhắn chưa đọc) theo hội thoại, từ đó tránh phát sinh N+1 queries khi cần tính toán trên nhiều hội thoại đồng thời. Tương tự, các báo cáo thống kê (reporting) áp dụng truy vấn thô cho các phép đếm và nhóm theo thời gian (ví dụ \texttt{date\_trunc}) nhằm tạo số liệu tổng hợp phục vụ dashboard quản trị.

\chapter{HIỆN THỰC HÓA HỆ THỐNG}

\section{Mô-đun AI Tutor theo kiến trúc RAG}
Phần này trình bày hiện thực mô-đun AI Tutor trong EduVerse theo hướng Retrieval-Augmented Generation (RAG). Trong phiên bản triển khai, hệ thống xây dựng pipeline nội bộ để kiểm soát chunking, indexing và truy vấn ngữ nghĩa trên cùng hạ tầng PostgreSQL.

\subsection{Chuẩn bị dữ liệu và chunking nội dung bài học}
Nguồn tri thức của AI Tutor là dữ liệu bài học (lesson) do giáo viên biên soạn. Trước khi tạo embedding, hệ thống ghép nội dung theo dạng \texttt{\# <title>\textbackslash n\textbackslash n<content>} và đưa qua bước chunking với tham số \texttt{maxChars} (mặc định 1200). Thuật toán ưu tiên tách theo đoạn (phân cách bởi \texttt{\textbackslash n\textbackslash n}); nếu một đoạn vẫn vượt ngưỡng, hệ thống fallback sang tách theo từ để đảm bảo mỗi chunk không vượt quá giới hạn, qua đó cân bằng giữa độ phủ ngữ cảnh và chi phí tính toán embedding.

\begin{figure}[H]
    \centering
    \IfFileExists{figures/uml/flow-rag.png}{
        \includegraphics[width=0.95\textwidth,height=0.8\textheight,keepaspectratio]{figures/uml/flow-rag.png}
    }{
        \IfFileExists{docs/luanvan/figures/uml/flow-rag.png}{
            \includegraphics[width=0.95\textwidth,height=0.8\textheight,keepaspectratio]{docs/luanvan/figures/uml/flow-rag.png}
        }{
            \framebox{\parbox{0.95\textwidth}{\centering\vspace{1cm}Activity/Flow Diagram (PlantUML PNG)\vspace{1cm}}}
        }
    }
    \caption{Quy trình xử lý AI Tutor theo kiến trúc RAG}
    \label{fig:flow-rag}
\end{figure}

\subsection{Tạo embedding bằng Gemini Embedding}
Đối với mỗi chunk, hệ thống gọi Gemini Embedding model \texttt{gemini-embedding-001} \cite{gemini} để tạo vector biểu diễn ngữ nghĩa. Embedding được tiêu chuẩn hóa về số chiều \texttt{1536}. Trong hiện thực, chunk bài học sử dụng task type \texttt{RETRIEVAL\_DOCUMENT} và câu hỏi học sinh sử dụng \texttt{RETRIEVAL\_QUERY} để tối ưu truy vấn.

Hệ thống tăng độ tin cậy (reliability) khi gọi dịch vụ bên ngoài bằng cách phân loại một số lỗi tạm thời (ví dụ HTTP 429/rate limit, timeout, 502/503) vào nhóm \textit{retryable errors} và thực hiện retry theo cơ chế \textbf{exponential backoff} (thời gian chờ tăng theo lũy thừa và có ngưỡng trần), nhằm giảm nguy cơ tạo tải đột biến và tuân thủ hạn mức dịch vụ. Đồng thời, hệ thống kiểm tra ràng buộc số chiều embedding (\texttt{1536}) để tránh ghi dữ liệu sai định dạng vào cột vector.

\subsection{Lưu trữ embedding và truy vấn vector similarity bằng pgvector}
Embedding sau khi tạo được lưu vào bảng \texttt{lesson\_embedding\_chunks}. Mỗi chunk được định danh ổn định theo cặp khóa \texttt{(lessonId, chunkIndex)}. Để đảm bảo \textbf{tính lũy đẳng (idempotency)} của pipeline indexing, hệ thống tính \texttt{contentHash} (SHA-256) cho từng chunk và bỏ qua việc tạo embedding nếu hash không đổi. Khi cần ghi dữ liệu, hệ thống thực hiện upsert bằng câu lệnh \texttt{INSERT ... ON CONFLICT (lessonId, chunkIndex) DO UPDATE}, nhờ đó chạy lại nhiều lần vẫn không tạo trùng bản ghi mà chỉ cập nhật khi nội dung thay đổi.

\begin{lstlisting}[language=JavaScript,caption={Cơ chế contentHash và upsert trong pipeline indexing (trích từ src/lib/rag/indexLessonEmbeddings.ts)},label={lst:code-contenthash-upsert}]
function sha256Hex(text: string): string {
  return crypto.createHash("sha256").update(text, "utf8").digest("hex");
}

const contentHash = sha256Hex(ch.content);
const vec = toVectorLiteral(embedding);
const id = `lec_${lessonId}_${ch.index}`;

await prisma.$executeRaw`
  INSERT INTO "lesson_embedding_chunks" (
    "id", "lessonId", "courseId", "chunkIndex", "content", "contentHash", "embedding", "updatedAt"
  )
  VALUES (
    ${id}, ${lessonId}, ${courseId}, ${ch.index}, ${ch.content}, ${contentHash}, ${vec}::vector, NOW()
  )
  ON CONFLICT ("lessonId", "chunkIndex")
  DO UPDATE SET
    "content" = EXCLUDED."content",
    "contentHash" = EXCLUDED."contentHash",
    "embedding" = EXCLUDED."embedding",
    "updatedAt" = NOW();
`;
\end{lstlisting}

Ngoài ra, hệ thống có cơ chế \textbf{garbage collection} cho dữ liệu embedding: khi nội dung bài giảng bị rút ngắn làm giảm số lượng chunk, các chunk ``dư thừa'' có \texttt{chunkIndex} lớn hơn chunk cuối cùng sẽ được xóa khỏi bảng embedding. Cách làm này giúp đảm bảo dữ liệu vector trong DB luôn phản ánh đúng phiên bản nội dung mới nhất và tránh gây nhiễu khi truy vấn Top-$k$.

Khi học sinh đặt câu hỏi, hệ thống:
\begin{enumerate}
    \item tạo embedding cho câu hỏi;
    \item truy vấn các chunk gần nhất theo khoảng cách vector trong PostgreSQL;
    \item ghép các chunk Top-$k$ vào prompt làm ngữ cảnh trả lời.
\end{enumerate}

Truy vấn similarity sử dụng toán tử khoảng cách của pgvector dưới dạng \texttt{("embedding" <=> queryVector::vector) AS distance}, sau đó sắp xếp \texttt{ORDER BY distance ASC} và lấy \texttt{LIMIT topK}. Kết quả là danh sách các đoạn nguồn được dùng để ràng buộc câu trả lời của AI Tutor, nhằm giảm hiện tượng “ảo giác” \cite{rag}.

\subsection{Tổ chức pipeline indexing (teacher-trigger và cron)}
Hệ thống hỗ trợ hai cách kích hoạt indexing:
\begin{itemize}

    \item \textbf{Teacher-trigger:} giáo viên gọi endpoint indexing theo khóa học để tạo hoặc làm mới embedding khi cập nhật lesson.
    \item \textbf{Cron indexing:} tiến trình định kỳ quét lesson cập nhật và thực hiện indexing, có xác thực bằng bí mật cron.
\end{itemize}

\section{Hệ thống đánh giá và giám sát thi cử}
\subsection{Cơ chế tự động chấm bài}
Đối với bài tự luận (ESSAY), hệ thống hỗ trợ giáo viên bằng cơ chế gợi ý chấm điểm: server lấy nội dung bài nộp, kết hợp thông tin đề bài và rubric, sau đó gọi Gemini để sinh gợi ý điểm số và nhận xét. Các API AI được áp dụng giới hạn tần suất (rate limit) nhằm giảm nguy cơ lạm dụng tài nguyên dịch vụ AI.

\begin{figure}[H]
    \centering
    \IfFileExists{figures/uml/seq-file-submission.png}{
        \includegraphics[width=0.95\textwidth,height=0.8\textheight,keepaspectratio]{figures/uml/seq-file-submission.png}
    }{
        \IfFileExists{docs/luanvan/figures/uml/seq-file-submission.png}{
            \includegraphics[width=0.95\textwidth,height=0.8\textheight,keepaspectratio]{docs/luanvan/figures/uml/seq-file-submission.png}
        }{
            \framebox{\parbox{0.95\textwidth}{\centering\vspace{1cm}Sequence Diagram (PlantUML PNG)\vspace{1cm}}}
        }
    }
    \caption{Quy trình nộp bài dạng tệp}
    \label{fig:seq-file-submission}
\end{figure}

\subsection{Phát hiện hành vi đáng ngờ và tổng hợp báo cáo chống gian lận (Anti-cheat)}
Trong EduVerse, dữ liệu chống gian lận được thu thập dưới dạng \textbf{event log} trong quá trình học sinh làm bài quiz. Các sự kiện được ghi nhận tại giao diện thi (client) và lưu về server để phục vụ phân tích.

\subsubsection{Thu thập dữ liệu: ánh xạ hành vi trình duyệt sang exam events}
Các tín hiệu gian lận trọng yếu trong hiện thực bao gồm:
\begin{itemize}

    \item \textbf{Chuyển tab (visibilitychange):} khi \texttt{document.hidden = true}, hệ thống ghi event \texttt{TAB\_SWITCH\_DETECTED} với nguồn \texttt{visibilitychange}.
    \item \textbf{Mất focus cửa sổ (blur):} khi \texttt{window.blur} xảy ra, hệ thống có thể ghi event \texttt{TAB\_SWITCH\_DETECTED} với nguồn \texttt{window\_blur}.
    \item \textbf{Rời cửa sổ (WINDOW\_BLUR):} trong một số luồng giao diện thi khác, client có thể ghi trực tiếp event \texttt{WINDOW\_BLUR} khi phát hiện \texttt{window.blur}. Tín hiệu này phản ánh hành vi chuyển cửa sổ/ứng dụng trong lúc làm bài.
    \item \textbf{Clipboard/context menu:} hệ thống chặn thao tác copy/paste và ghi event \texttt{COPY\_PASTE\_ATTEMPT} với nguồn \texttt{contextmenu} hoặc phím tắt tương ứng.
    \item \textbf{Thoát fullscreen:} khi bắt buộc fullscreen và học sinh thoát fullscreen, hệ thống ghi event \texttt{FULLSCREEN\_EXIT}.
\end{itemize}

Để đảm bảo \textbf{tính nhất quán} khi có nhiều biến thể ghi log từ client, server thực hiện bước \textbf{chuẩn hóa (normalization)} \texttt{eventType} trước khi tính điểm. Cụ thể, các event \texttt{TAB\_SWITCH\_DETECTED} (dù được phát sinh từ \texttt{visibilitychange} hay \texttt{window\_blur}) đều được gom về nhóm \texttt{TAB\_SWITCH}. Tương tự, \texttt{COPY\_PASTE\_ATTEMPT} được chuẩn hóa về nhóm \texttt{CLIPBOARD}. Việc chuẩn hóa này giúp công thức tính \texttt{suspicionScore} không phụ thuộc vào chi tiết hiện thực UI/trình duyệt, đồng thời tạo điều kiện so sánh điểm số ổn định theo thời gian.

\begin{lstlisting}[language=JavaScript,caption={Chuẩn hóa sự kiện và tính điểm suspicionScore (trích từ src/lib/exam-session/antiCheatScoring.ts)},label={lst:code-anti-cheat-score}]
function normalizeEventType(eventType: string): string {
  const raw = (eventType || "").toString().trim();
  if (!raw) return "";

  switch (raw) {
    case "TAB_SWITCH_DETECTED":
      return "TAB_SWITCH";
    case "COPY_PASTE_ATTEMPT":
      return "CLIPBOARD";
    default:
      return raw;
  }
}

export function computeQuizAntiCheatScore(events: ExamEventForScoring[]): AntiCheatScoreResult {
  const countsByType: Record<string, number> = {};

  for (const ev of events) {
    const type = normalizeEventType(ev.eventType);
    if (!type) continue;
    countsByType[type] = (countsByType[type] ?? 0) + 1;
  }

  // ... rule-based breakdown + clamp score 0..100
}
\end{lstlisting}

\subsubsection{Thuật toán tính điểm: suspicionScore theo rule-based scoring}
Server tính điểm nghi ngờ \texttt{suspicionScore} theo cơ chế rule-based, ánh xạ từng loại event sang một rule với trọng số và mức trần điểm. Mỗi rule có dạng:
\begin{quote}
\texttt{points = min(maxPoints, count * pointsPerHit)}
\end{quote}
Điểm tổng được chặn trong khoảng $[0,100]$. Trong hiện thực, các rule trọng yếu và tham số tương ứng được tổng hợp trong Bảng \ref{tab:anti-cheat-weights}.

\begin{table}[H]
\centering
\small
\begin{tabularx}{\textwidth}{|X|>{\centering\arraybackslash}p{3cm}|>{\centering\arraybackslash}p{3cm}|}
\hline
\rowcolor[gray]{0.9} \textbf{Quy tắc / Nhóm sự kiện (sau chuẩn hóa)} & \textbf{pointsPerHit} & \textbf{maxPoints} \\ \hline
Thoát toàn màn hình (\texttt{FULLSCREEN\_EXIT}) & 20 & 40 \\ \hline
Chuyển tab trình duyệt (\texttt{TAB\_SWITCH}) & 12 & 60 \\ \hline
Rời khỏi cửa sổ (\texttt{WINDOW\_BLUR}) & 5 & 20 \\ \hline
Thao tác Clipboard (\texttt{CLIPBOARD}) & 8 & 24 \\ \hline
Phím tắt đáng ngờ (\texttt{SHORTCUT}) & 6 & 18 \\ \hline
\end{tabularx}
\caption{Bảng trọng số tính điểm \texttt{suspicionScore} theo cơ chế rule-based}
\label{tab:anti-cheat-weights}
\end{table}
 
 \begin{figure}[H]
     \centering
     \IfFileExists{docs/luanvan/figures/uml/flow-anti-cheat.png}{
         \includegraphics[width=0.95\textwidth,height=0.8\textheight,keepaspectratio]{docs/luanvan/figures/uml/flow-anti-cheat.png}
     }{
         \framebox{\parbox{0.95\textwidth}{\centering\vspace{1cm}Activity/Flow Diagram (PlantUML PNG)\vspace{1cm}}}
     }
     \caption{Luồng xử lý chống gian lận: scoring và AI summary}
     \label{fig:flow-anti-cheat}
 \end{figure}
 
 \chapter{ĐÁNH GIÁ VÀ THỬ NGHIỆM}
 
 \section{Môi trường thử nghiệm và phương pháp đánh giá}
 Hệ thống EduVerse được kiểm thử theo hướng kết hợp giữa:
 \begin{itemize}
     \item \textbf{Kiểm thử đơn vị (Unit Test):} sử dụng Vitest (script \texttt{vitest run}) để kiểm tra các hàm nghiệp vụ quan trọng như chia đoạn văn bản cho RAG (\texttt{chunkText}) và tính điểm nghi ngờ anti-cheat (\texttt{suspicionScore}).
     \item \textbf{Đánh giá luồng API:} đánh giá luồng xử lý của các endpoint chính (RAG Tutor chat, ghi exam events, anti-cheat scoring và AI summary) dựa trên ràng buộc dữ liệu và điều kiện xác thực/giới hạn tần suất.
     \item \textbf{Đánh giá trải nghiệm người dùng (UX):} tập trung vào tính nhất quán của luồng thao tác theo vai trò và cơ chế phản hồi trạng thái ở các thao tác quan trọng.
 \end{itemize}
 
 \section{Kiểm thử chức năng}
 Bảng \ref{tab:testcases-core} tổng hợp 5 kịch bản kiểm thử chức năng quan trọng, tập trung vào các luồng có rủi ro cao trong vận hành thực tế (xác thực theo vai trò, nộp bài, AI Tutor và chống gian lận).
 
\begin{table}[H]
\centering
\small
\begingroup
\setlength{\tabcolsep}{3pt}
\renewcommand{\arraystretch}{1.15}
\begin{tabularx}{\textwidth}{|p{0.9cm}|p{2.8cm}|X|X|X|}
\hline
\textbf{TC} & \textbf{Chức năng} & \textbf{Tiền điều kiện} & \textbf{Các bước thực hiện (tóm tắt)} & \textbf{Kỳ vọng} \\
\hline
TC01 & Xác thực portal theo vai trò (RBAC) & Có tài khoản Teacher/Student; phiên đăng nhập hợp lệ & (1) Đăng nhập. (2) Gọi API/đi tới trang thuộc vai trò khác. & Bị từ chối với HTTP 403; chỉ role hợp lệ được phép truy cập. \\
\hline
TC02 & Nộp bài tập (AssignmentSubmission) & Student thuộc Classroom có Assignment; attempt hợp lệ & (1) Student gửi bài nộp. (2) Kiểm tra DB có bản ghi \texttt{(assignmentId, studentId, attempt)}. & Tạo bản nộp thành công; ràng buộc unique tránh trùng attempt. \\
\hline
TC03 & AI Tutor phản hồi theo dữ liệu RAG & Student là thành viên lớp; lesson đã/hoặc chưa index embedding & (1) Gửi POST . (2) Quan sát câu trả lời và danh sách nguồn. & Nếu có embedding: trả về \texttt{answer + sources}. Nếu chưa có embedding: trả về thông báo và cờ \texttt{noEmbeddings}. \\
\hline
TC04 & Anti-cheat ghi nhận log sự kiện & Student đang làm Quiz; có \texttt{assignmentId} & (1) Client phát hiện sự kiện (tab switch/fullscreen exit). (2) Gửi POST /api/exam-events. & DB tạo bản ghi \texttt{exam\_events}; metadata bị giới hạn kích thước. \\
\hline
TC05 & Giáo viên xem điểm nghi ngờ và AI summary & Teacher là chủ Assignment (Quiz); có dữ liệu exam events & (1) Teacher xem điểm nghi ngờ. (2) Gọi /api/ai/anti-cheat/summary. & Trả về \texttt{suspicionScore}, \texttt{riskLevel}; AI summary chỉ áp dụng cho Quiz và có rate-limit. \\
\hline
\end{tabularx}
\endgroup
\caption{Các kịch bản kiểm thử chức năng cốt lõi của EduVerse}
\label{tab:testcases-core}
\end{table}

\section{Đánh giá độ chính xác và hiệu năng của AI Tutor}
Luồng RAG Tutor (\texttt{/api/ai/tutor/chat}) gồm ba pha chính: (i) kiểm tra xác thực và thành viên lớp, (ii) tạo embedding truy vấn và truy vấn pgvector để lấy Top-$k$ chunk gần nhất, (iii) gọi mô hình sinh nội dung để tạo câu trả lời có ràng buộc nguồn tham khảo.

Về kiểm soát tải và độ ổn định, endpoint áp dụng \textbf{rate-limit hai lớp}:
\begin{itemize}
    \item Theo IP: giới hạn 20 yêu cầu / 10 phút.
    \item Theo người dùng (student): giới hạn 20 yêu cầu / 10 phút.
\end{itemize}
Khi vượt ngưỡng, hệ thống trả về HTTP 429 kèm \texttt{Retry-After}. Ngoài ra, hệ thống giới hạn kích thước message và lịch sử hội thoại để giảm chi phí, đồng thời giới hạn độ dài phản hồi (\texttt{maxOutputTokens}) nhằm ổn định độ trễ khi số lượng người dùng tăng.

\section{Đánh giá anti-cheat: scoring và AI summary}
Đối với chống gian lận, hệ thống tách hai tầng:
\begin{itemize}
    \item \textbf{Tầng rule-based scoring:} tính \texttt{suspicionScore} trong khoảng $[0,100]$ dựa trên tần suất sự kiện và cơ chế \textit{cap điểm theo từng quy tắc}. Ví dụ: \texttt{FULLSCREEN\_EXIT} có 20 điểm/lần nhưng tối đa 40 điểm; \texttt{TAB\_SWITCH} có 12 điểm/lần tối đa 60 điểm.
    \item \textbf{Tầng AI summary:} endpoint \texttt{/api/ai/anti-cheat/summary} chỉ áp dụng cho Quiz và lấy tối đa 250 sự kiện theo thời gian, sau đó tạo tóm tắt dựa trên \texttt{suspicionScore}, \texttt{riskLevel} và breakdown. Endpoint áp dụng rate-limit theo IP và theo teacher nhằm đảm bảo chi phí và độ ổn định.
\end{itemize}
Thiết kế hai tầng giúp hệ thống giữ được tính quyết định của điểm nghi ngờ (phục vụ so sánh/đối chiếu) đồng thời dùng AI để tăng khả năng diễn giải bằng ngôn ngữ tự nhiên.

\section{Đánh giá trải nghiệm người dùng (UI/UX)}
EduVerse tổ chức giao diện theo portal đa vai trò (Teacher/Student/Parent/Admin) và áp dụng cơ chế phản hồi UI nhất quán.

Ở Teacher dashboard, dữ liệu được tải qua cơ chế caching/revalidate và có \textbf{skeleton loading} khi chưa có dữ liệu, \textbf{error state} rõ ràng khi lỗi và \textbf{empty state} có hướng dẫn hành động.

Ở giao diện chat của học sinh, hệ thống có cơ chế xử lý rate-limit thân thiện (hiển thị dialog kèm thời gian chờ) và có bước chuẩn hóa nội dung câu trả lời để tránh lộ thông tin kỹ thuật. Ngoài ra, hội thoại được lưu theo bài học để đảm bảo tính liên tục khi người dùng quay lại.

\section{Kết quả thực nghiệm}
Phần này trình bày kết quả đo thực nghiệm ở mức hệ thống (end-to-end) theo ba kịch bản chính: S1 (AI Tutor chat), S2 (ghi \texttt{ExamEvent}) và S3 (AI Anti-cheat Summary). Dữ liệu được thu thập bằng cơ chế theo dõi hiệu năng ở phía server và xuất ra theo khoảng thời gian 60 phút.

Bảng \ref{tab:experiment-results} tổng hợp các chỉ số đại diện theo từng endpoint: số request, độ trễ trung bình, P95, độ trễ lớn nhất và tỷ lệ lỗi.

\begin{table}[H]
\centering
\small
\begingroup
\setlength{\tabcolsep}{3pt}
\renewcommand{\arraystretch}{1.15}
\begin{tabularx}{\textwidth}{|p{4.2cm}|X|p{3.2cm}|}
\hline
\textbf{Hạng mục} & \textbf{Mô tả đo lường} & \textbf{Kết quả (đo thực nghiệm)} \\ \hline
API RAG Tutor  & Thời gian phản hồi end-to-end từ lúc gửi câu hỏi đến lúc nhận câu trả lời; bao gồm tạo embedding truy vấn, truy vấn pgvector Top-$k$ và sinh nội dung. & n=15; Avg 11.8s; P95 13.1s; Max 13.1s; Error 0\% \\ \hline
Thu thập \texttt{ExamEvent}  & Độ trễ ghi nhận sự kiện làm bài (server-side) và tỷ lệ request ghi log thành công. & n=300; Avg 5.9s; P95 7.6s; Max 10.7s; Error 0\% \\ \hline
AI Anti-cheat Summary  & Thời gian sinh tóm tắt chống gian lận dựa trên \texttt{suspicionScore}/\texttt{riskLevel} và tối đa 250 sự kiện theo attempt. & n=8; Avg 10.5s; P95 12.1s; Max 12.1s; Error 0\% \\ \hline
\end{tabularx}
\endgroup
\caption{Tóm tắt kết quả đo thực nghiệm (end-to-end) của hệ thống EduVerse}
\label{tab:experiment-results}
\end{table}
\noindent Trong kịch bản S2 (thu thập \texttt{ExamEvent}), hệ thống ghi nhận 300/300 yêu cầu ghi log thành công, tương ứng successRate xấp xỉ 100\% trong lần đo.

\section{Hạn chế}
Mặc dù hệ thống đã đáp ứng các luồng nghiệp vụ cốt lõi, EduVerse vẫn tồn tại một số hạn chế cần được xem xét khi triển khai thực tế.

\textbf{(1) Độ trễ phụ thuộc dịch vụ AI bên ngoài.} Các tác vụ tạo embedding, sinh câu trả lời RAG, tạo tóm tắt anti-cheat và gợi ý chấm bài đều phụ thuộc Gemini API, do đó độ trễ end-to-end có thể dao động theo chất lượng mạng, tải hệ thống phía nhà cung cấp và hạn mức (rate limit). Điều này ảnh hưởng trực tiếp đến trải nghiệm tương tác thời gian thực của học sinh và tốc độ xử lý nghiệp vụ của giáo viên.

\textbf{(2) Chi phí token và chi phí vận hành.} Kiến trúc RAG phát sinh chi phí ở cả hai pha: indexing (tạo embedding cho học liệu) và inference (sinh nội dung theo ngữ cảnh). Mặc dù pipeline đã tối ưu bằng cơ chế \texttt{contentHash} để tránh index lại nội dung không đổi, chi phí tổng vẫn có thể tăng đáng kể khi số lượng lớp học, bài giảng và truy vấn tăng.

\textbf{(3) Phụ thuộc chất lượng học liệu đầu vào.} RAG giúp bám sát tài liệu nhưng chất lượng phản hồi phụ thuộc trực tiếp vào cấu trúc và độ rõ ràng của học liệu. Nếu nội dung bài giảng thiếu mạch lạc hoặc có thông tin sai lệch, các đoạn truy xuất có thể không phù hợp, làm giảm chất lượng câu trả lời.

\textbf{(4) Giới hạn của anti-cheat dựa trên event log.} Cơ chế scoring hiện tại chủ yếu dựa trên tín hiệu trình duyệt và các quy tắc (rule-based), do đó vẫn có nguy cơ false positive/false negative. Bản tóm tắt AI hỗ trợ diễn giải nhưng không thay thế hoàn toàn vai trò ra quyết định của giáo viên và quy chế của nhà trường.

% ======================================================
% CHƯƠNG 6: KẾT LUẬN VÀ HƯỚNG PHÁT TRIỂN
% ======================================================

\chapter{KẾT LUẬN VÀ HƯỚNG PHÁT TRIỂN}

\section{Kết quả đạt được}
Luận văn đã nghiên cứu và hiện thực hóa hệ thống \textbf{EduVerse} theo định hướng Smart LMS đa vai trò, trong đó trọng tâm là tích hợp AI theo kiến trúc RAG và tăng cường giám sát liêm chính học thuật.

Các kết quả đạt được có thể tóm tắt như sau:
\begin{itemize}
    \item \textbf{Kiến trúc triển khai full-stack thống nhất:} ứng dụng được hiện thực hóa bằng Next.js App Router theo mô hình tổ chức UI và API trong cùng codebase, giúp các tác vụ nhạy cảm (truy vấn CSDL, gọi Gemini API, xử lý phân quyền) được thực thi phía server.
    \item \textbf{Xác thực và phân quyền theo RBAC:} hệ thống sử dụng NextAuth (JWT strategy) và middleware để đảm bảo người dùng truy cập đúng portal theo vai trò, đồng thời các API Routes kiểm tra quyền sở hữu dữ liệu và vai trò trước khi xử lý.
    \item \textbf{AI Tutor theo kiến trúc RAG:} xây dựng pipeline chunking, indexing và truy vấn vector trên PostgreSQL + pgvector; sử dụng \texttt{gemini-embedding-001} cho embedding và \texttt{gemini-2.5-flash-lite} cho sinh câu trả lời có ràng buộc ngữ cảnh.
    \item \textbf{Tối ưu indexing theo hướng lũy đẳng:} cơ chế \texttt{contentHash} kết hợp upsert giúp tránh tạo embedding lại khi nội dung không đổi, góp phần giảm chi phí và tăng tính ổn định khi vận hành.
    \item \textbf{Giám sát thi và chống gian lận:} thu thập event log, chuẩn hóa \texttt{eventType}, tính \texttt{suspicionScore} theo rule-based scoring và cung cấp tóm tắt diễn giải bằng AI để hỗ trợ giáo viên rà soát nhanh.
    \item \textbf{Hỗ trợ chấm bài tự luận:} hệ thống cung cấp gợi ý chấm điểm/nhận xét dựa trên bài nộp và rubric, giúp giảm tải các tác vụ lặp lại cho giáo viên.
\end{itemize}

Các kết quả trên cho thấy hướng tiếp cận tích hợp RAG và LLM vào hệ thống LMS theo hướng ưu tiên tính kiểm chứng, khả năng truy vết và kiểm soát chi phí. Các kết quả đạt được là nền tảng cho việc tiếp tục nghiên cứu, hoàn thiện và triển khai trong bối cảnh giáo dục thực tế.

\section{Đối chiếu với mục tiêu nghiên cứu}
So với các mục tiêu đề ra ở Chương 1, hệ thống đáp ứng được các yêu cầu chính:
\begin{itemize}
    \item \textbf{Kiến trúc LMS đa vai trò:} mô hình multi-portal theo RBAC giúp phân tách luồng thao tác theo nhóm người dùng và giảm nguy cơ truy cập nhầm chức năng.
    \item \textbf{AI Tutor dựa trên CSDL vector:} pipeline RAG được hiện thực hóa đầy đủ từ indexing học liệu đến truy vấn Top-$k$ và sinh câu trả lời, đồng thời có cơ chế kiểm soát truy cập theo lesson/course.
    \item \textbf{Tự động hóa đánh giá:} cơ chế gợi ý chấm bài tự luận hỗ trợ giáo viên tăng tốc phản hồi, có thể mở rộng theo hướng bán tự động với kiểm soát của con người (human-in-the-loop).
    \item \textbf{Liêm chính học thuật:} scoring dựa trên event log kết hợp AI summary hỗ trợ phát hiện/diễn giải hành vi đáng ngờ, nhưng vẫn cần kết hợp quy chế và quyết định của giáo viên.
\end{itemize}

\section{Hướng phát triển tương lai}
Trong tương lai, hệ thống có thể được nâng cấp theo các hướng sau:
\begin{itemize}
    \item \textbf{Multimodal AI Tutor:} mở rộng khả năng hiểu và trả lời dựa trên hình ảnh/biểu đồ, hoặc tài liệu PDF có cấu trúc, giúp hỗ trợ tốt hơn các môn học có nội dung trực quan.
    \item \textbf{Proctoring nâng cao:} kết hợp thêm tín hiệu ngoài trình duyệt (ví dụ video/âm thanh) khi có cơ chế đồng thuận và chính sách bảo mật rõ ràng, nhằm giảm các trường hợp gian lận ngoài phạm vi quan sát của event log.
    \item \textbf{Tối ưu chi phí và độ trễ:} áp dụng cache theo câu hỏi phổ biến, tóm tắt ngữ cảnh theo phiên, điều chỉnh Top-$k$ động và giám sát chi phí token theo lớp/môn để tối ưu vận hành.
    \item \textbf{Đánh giá chất lượng AI có hệ thống:} xây dựng bộ benchmark theo từng môn và các tiêu chí faithfulness/groundedness để theo dõi chất lượng câu trả lời RAG theo thời gian.
    \item \textbf{Mở rộng trải nghiệm đa nền tảng:} phát triển ứng dụng di động, thông báo đẩy và cá nhân hóa lộ trình học dựa trên tiến độ và lịch sử tương tác.
\end{itemize}

\section{Kết luận}
EduVerse chứng minh tính khả thi của việc tích hợp RAG và LLM vào hệ thống LMS theo hướng ưu tiên tính kiểm chứng, khả năng truy vết và kiểm soát chi phí. Các kết quả đạt được là nền tảng cho việc tiếp tục nghiên cứu, hoàn thiện và triển khai trong bối cảnh giáo dục thực tế.

% ======================================================
% TÀI LIỆU THAM KHẢO
% ======================================================
\begin{thebibliography}{99}
\addcontentsline{toc}{chapter}{Tài liệu tham khảo}
\bibitem{nextjs} Vercel, "Next.js 14 Documentation," 2024. [Online]. Available: \url{https://nextjs.org/docs}.
\bibitem{nextauth} NextAuth.js, "NextAuth.js Documentation," 2024. [Online]. Available: \url{https://next-auth.js.org}.
\bibitem{prisma} Prisma, "Prisma Documentation," 2024. [Online]. Available: \url{https://www.prisma.io/docs}.
\bibitem{postgresql} The PostgreSQL Global Development Group, "PostgreSQL Documentation," 2024. [Online]. Available: \url{https://www.postgresql.org/docs/}.
\bibitem{pgvector} pgvector, "pgvector: Open-source vector similarity search for Postgres," 2024. [Online]. Available: \url{https://github.com/pgvector/pgvector}.
\bibitem{supabase} Supabase, "Supabase Documentation," 2024. [Online]. Available: \url{https://supabase.com/docs}.
\bibitem{plantuml} PlantUML, "PlantUML Documentation," 2024. [Online]. Available: \url{https://plantuml.com/}.
\bibitem{gemini} Google DeepMind, "Gemini: A Family of Highly Capable Multimodal Models," 2023.
\bibitem{rag} Lewis, P., et al., "Retrieval-Augmented Generation for Knowledge-Intensive NLP Tasks," NeurIPS, 2020.
\end{thebibliography}

\end{document}